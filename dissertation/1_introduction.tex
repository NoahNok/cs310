\chapter{Introduction}
\label{ch:introduction}
Within the understanding of how a modern processor operates, we often look into the underlying execution and physical operations performed. Often, being able to visualise the processes and ideas is far more beneficial \cite{quratulain_2019_a} to our understanding. Rather, than simply reading written texts such as manuals, papers, books or diagrams. A visualisation provides educational benefits, with visual aids helping to increase the understanding of complex material and reinforces meaning where written material lacks the ability to.

RISC-V \cite{waterman_2019_the} is an open source load-store instruction set architecture that has been gaining traction since its major release in 2015. However, despite being only 7 years old and currently being heavily explored, the range of educational tools available to visualise its physical operation are limited. With most tools for simulating just providing a basic output after emulation. Further, out of all the limited tools available, most are burdened with being unintuitive for users with no or limited knowledge of RISC-V.

Thus, this project aims to build up a Simulation tool (including Visualisation and Emulation) for RISC-V from the ground up. Providing a tool for individuals and educators to increase their understanding of RISC-V through a visual approach. Following the RISC-V specification \cite{waterman_2011_the} mostly to provide a rigid understanding, whilst keeping a simplified approach to avoid overburdening the end user. The project being a simulator over just an emulator means the project is able to provide a visual aspect, rather than just executing RISC-V code. This provides a better learning experience for those using the end application and a deeper understanding of RISC-V. The differences between a Simulator and a Emulator are discussed in Section \ref{sec:sim_vs_em}.

RISC-V provides a 32 bit base specification to use, with the support for additional extensions. These extensions exist so that RISC-V may incorporate more modern features to improve performance and increase capability. Some of these extensions include: Multiply and Divide, and Floating Point. An addition to the project would be to include some of these basic extensions within the final application to provide a wider range of simulation. This could be approached in multiple ways, one of which may be to directly hard-code them into the core application to provide a concrete and reliable implementation of extensions. Or, provide them as additional modules via a module system that can be loaded and unloaded to add/remove complexity. Of these two cases, the modular approach is the most appropriate, allowing for its mentioned benefits and ability to split up the whole project into smaller chunks.

