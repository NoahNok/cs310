\documentclass[a4paper,fleqn,twoside,12pt]{report}

%%%%%%%%%%%%%%%%%%%%

\input{common/common.tex}
%%%%%%%%%%%%%%%%%%%%%%%%%%%%%%%%%%%%%%%%%%%%%%%%%%%%%%%%%%%%%%%%%%%%%%%%%%%%%%%
%% Project-specific configuration
%%%%%%%%%%%%%%%%%%%%%%%%%%%%%%%%%%%%%%%%%%%%%%%%%%%%%%%%%%%%%%%%%%%%%%%%%%%%%%%

\author{Noah Hollowell}
\title{Visual RISC V Simulator}
% \supervisor{Your supervisor's name}
% \yearofstudy{3\textsuperscript{rd}}

%%%%%%%%%%%%%%%%%%%%%%%%%%%%%%%%%%%%%%%%%%%%%%%%%%%%%%%%%%%%%%%%%%%%%%%%%%%%%%%


\toggletrue{IsDissertation}

%%%%%%%%%%%%%%%%%%%%
\usepackage{enumitem}
\usepackage{acronym}
\usepackage{pdfpages}
\pagestyle{plain}
\renewcommand{\headrulewidth}{0.0pt}

\makeatletter
\fancypagestyle{plain}{
	\fancyhf{}
	\fancyhead[LE]{\thepage}
	\fancyhead[RE]{\textit{\@author}}
	\fancyhead[RO]{\thepage}
	\fancyhead[LO]{\textit{\@title}}
}
\makeatother

%%%%%%%%%%%%%%%%%%%%

\begin{document}

\input{common/titlepage.tex}

\pagestyle{plain}

%%%%% SWITCH TO BIG 0 SYMBOL

\begin{abstract}
    Visual simulation of computer architectures is hard to come by and often hard to follow. Older architecture have ample education aids available, whereas newer ones such as RISC-V lack them.
    \\\\
    A simulation should provide a healthy compromise between usability and understanding, providing a comprehensive look inside an architectures workings, without becoming convoluted.
    \\\\
    This project aims to produce a simple, yet informative simulator for the RISC-V architecture that is accessible to both amateurs and professionals alike, with a focus on the core aspects, whilst providing user extensibility.
    \\\\
    \textbf{Keywords:} RISC-V, Java, Simulator, Educational, Visualisation
\end{abstract}
\renewcommand*\abstractname{Acknowledgements}
\begin{abstract}
Firstly, I would like to thank Adam Chester for proposing the project concept, and secondly my supervisor Richard Kirk, for taking over so quickly and taking such an interest after his predecessor Adam left the department, providing constructive feedback to help keep the project heading in the right direction.

I would also like to thank my friends and course-mates for helping test and provide feedback on the final application, and further my parents for proof reading this report.
\end{abstract}

\renewcommand*\abstractname{Abbreviations}
\begin{abstract}
\begin{acronym}
\acro{ALU}{Arithmetic Logic Unit}
\acro{CISC}{Complex Instruction Set Computer}
\acro{CU}{Control Unit}
\acro{ID}{Instruction Decoder}
\acro{IR}{Instruction Register}
\acro{RISC}{Reduced Instruction Set Computer}
\acro{UI}{User Interface}
\acro{WYSIWYG}{What You See Is What You Get}
\end{acronym}
\end{abstract}

\tableofcontents

\chapter{Introduction}
\label{ch:introduction}
Within the understanding of how a modern processor operates, we often look into the underlying execution and physical operations performed. Often, being able to visualise the processes and ideas is far more beneficial \cite{quratulain_2019_a} to our understanding. Rather, than simply reading written texts such as manuals, papers, books or diagrams. A visualisation provides educational benefits, with visual aids helping to increase the understanding of complex material and reinforces meaning where written material lacks the ability to.

RISC-V \cite{waterman_2019_the} is an open source load-store instruction set architecture that has been gaining traction since its major release in 2015. However, despite being only 7 years old and currently being heavily explored, the range of educational tools available to visualise its physical operation are limited. With most tools for simulating just providing a basic output after emulation. Further, out of all the limited tools available, most are burdened with being unintuitive for users with no or limited knowledge of RISC-V.

Thus, this project aims to build up a Simulation tool (including Visualisation and Emulation) for RISC-V from the ground up. Providing a tool for individuals and educators to increase their understanding of RISC-V through a visual approach. Following the RISC-V specification \cite{waterman_2011_the} mostly to provide a rigid understanding, whilst keeping a simplified approach to avoid overburdening the end user. The project being a simulator over just an emulator means the project is able to provide a visual aspect, rather than just executing RISC-V code. This provides a better learning experience for those using the end application and a deeper understanding of RISC-V. The differences between a Simulator and a Emulator are discussed in Section \ref{sec:sim_vs_em}.

RISC-V provides a 32 bit base specification to use, with the support for additional extensions. These extensions exist so that RISC-V may incorporate more modern features to improve performance and increase capability. Some of these extensions include: Multiply and Divide, and Floating Point. An addition to the project would be to include some of these basic extensions within the final application to provide a wider range of simulation. This could be approached in multiple ways, one of which may be to directly hard-code them into the core application to provide a concrete and reliable implementation of extensions. Or, provide them as additional modules via a module system that can be loaded and unloaded to add/remove complexity. Of these two cases, the modular approach is the most appropriate, allowing for its mentioned benefits and ability to split up the whole project into smaller chunks.


\chapter{Research}
\label{ch:research}

For the project to be successful, research needs to be done on a range of topics including most notably RISC-V, as-well as other considered architectures, implementation languages and existing solutions. These are all discussed in the following sections, but first we must quickly discuss the difference between a Simulator and Emulator to avoid any confusion.

\section{Emulator vs Simulator}\label{sec:sim_vs_em}
It is important to establish the difference between a \textbf{Emulator} and a \textbf{Simulator}, as both are referenced through this report. With a \textbf{Simulator} in essence encapsulating a \textbf{Emulator}.

\subsection{Emulator}
An Emulator aims to reproduce and mimic its target, in our case we will emulate RISC-V code as if it were being run on a physical RISC-V chip.

For example, older games are emulated to be run on newer hardware due to the original being hard to obtain or missing. Only the game is recreated with nothing else included.
\subsection{Simulator}
A Simulator aims to model the target, in this case it may include emulation of the target and then model on top of this emulation.

For example, many Flight Simulators exist. They emulate the operation of a plane, and include all the details with how to interact with it, from physical buttons to 3D rendering to represent what happens in reality.

\section{Architecture}
The project itself could of been produced for any architecture, thus it was important to consider a few different architectures and the pros and cons of each when producing a simulator. The project settled on one of RISC-V \cite{waterman_2019_the} , x86 \cite{intelcorporation_2023_intel} and ARM \cite{armltd_2023_defining}, which are discussed below, with RISC-V being chosen in the end thanks to its open-source nature with ample implementation material online, and being relatively current such that the end project will be useful in the indefinite future as both an education tool and a exploration tool..

\subsection{RISC-V}
RISC-V \cite{waterman_2019_the} is a little-endian load-store architecture. RISC-V began as a project in 2010 at the University of California, becoming officially introduced in 2015, with an aim "to develop revolutionary approaches as well as the technologies and techniques to provide power efficiency to enabled embedded computing systems" \cite{riscvinternational_2018_history} based on the \ac{RISC} design.

RISC-V is a \ac{RISC} processor with a load-store design. RISC-V splits instructions are split into two distinct sections: Memory Access and \ac{ALU} operations. As mentioned above, RISC-V is little-endian based, meaning that it stores the most significant bit of any value at the largest memory address associated with the value.

RISC-V consists of 32 base registers, of which each are 32 bits wide each. It follows a 4 byte aligned memory model, with the option to allow unaligned writing based on the respective implementation. RISC-V also supports variable length extensions, such that we are not limited to the base 32 bit width of registers, and are permitted to drop down to 16 bit or go beyond. This allows RISC-V to provide a suite of extension instruction sets such as: Single, Double and Quad Precision Floating Point; Multiply and Divide; and 64 and 128 Bit Integer support.

Thanks, to RISC-V's simple, yet extensible design, it makes it a good fit for the project, with a reduced base instruction set to implement as a solid core, followed by the ability to extend with optional instruction sets in the future.

\subsection{x86}
Alternatively, x86 was also considered briefly for the project. x86 \cite{intelcorporation_2023_intel} was created in 1978 as a 16 bit little-endian instruction set, with it now commonly being used with its 64 bit version produced in 2003. However, x86 is a \ac{CISC}, as a result the very basic 8086 \cite{amd_1989_8086} instruction set has around 81 unique instructions compared to RISC-V's 47, with each \ac{CISC} instruction able to perform multiple lower-level tasks compared to a \ac{RISC} instruction performing just its designated task.

Depending on the implementation of x86, the processor can have 6, 8 or 16 general purpose registers supporting the respective 16, 32 and 64 bit width values. However, since x86 \cite{intelcorporation_2023_intel} is so well known it host a wide range of extensions, with 38 currently available, such as Advanced Vector Encoding for 256 bit support and the FMA4 extension to allow 4-operand fused multiply add instructions.

Unfortunately, x86 is not completely open-source, with some advanced aspects requiring a licence from Intel or AMD for x86-64 specific implementations. Further, a wealth of simulators, emulators and interpreters already exist such as:
\begin{itemize}
    \item QEMU \cite{bellard_2023_qemu} which enables full simulation of operating systems, programs and even the option to run virtual machines with near native performance,
    \item DOSBox \cite{dosbox_2021_dosbox} is another, focusing more on emulating video games rather than operating systems, allowing for older MS-DOS games to be played on modern hardware., however, its last stable release was in 2019, with no recent updates as of 2021,
    \item Rosetta \cite{appleinc_2023_about} which was developed by Apple Inc to allow for compatibility between different instruction set architectures made explicitly for macOS. Unlike QEMU \cite{bellard_2023_qemu} and DOSBox \cite{dosbox_2021_dosbox}, Rosetta is specific to macOS and cannot be downloaded and run on any device, limiting its outreach and usability.
\end{itemize}



As a result of x86's CISC design and 81 base instructions, the decision was made to not implement x86 as the overhead of planning and implementing each instruction carefully and accurately would of taken up too much time, limiting the ability to produce a robust visualisation later that is simple to follow and understand.
\subsection{ARM}
Like RISC-V, ARM \cite{armltd_2023_defining}, is also a \ac{RISC} processor. The Advanced RISC Machines architecture (referred to as ARM processors) began in 1990 as a joint venture between Acorn Computers, Apple and VLSI Technology. More recently ARM has become more well known again due to its rise in use in portable devices like phones and laptops, thanks to its low power consumption, cost and high performance. \cite{schmitt_2021_is}

In terms of simulation, the 32 bit implementation started with the Intel 80386 \cite{intel_1998_intel386} consisting of 15$\times$ 32 bit general purpose registers, with the more modern 64 bit  processor such as the Core 2 Duo \cite{intel_2007_intel} supporting 31 64 bit general purpose registers, with both having support for up to 32 floating point registers, and like RISC-V it is little-endian, with around 34 discrete instructions, being far less than x86's 81.

ARM was not selected in the end due to it's new architectures being proprietary, thus anyone wishing to design a newer ARM processor must pay a licensing fee of around \$1-10 million and royalties ranging from 1-2\% of a chips sell price \cite{strategyzerag_2014_arm}. Whilst this wouldn't directly impact the project, it is preferable to work with a fully open-source architecture with ample implementation details available online, free to use forever. We may of considered using and older ARM design which is freely available, however simulating something older and less commonly used would provide little real-world benefit. Further, despite there being only 34 discrete instructions, they all have many conditional parts allowing them to perform different operations. This means that when implemented it corresponds to over 100 unique instructions. Thus, if implemented would require a significant amount on time and require instructions to be selectively cut to ensure the project stays on track. Thus ARM was not selected for the project and RISC-V was chosen.

\section{Implementation Language}
In order to produce a simulation of RISC-V, a suitable language must be chose that can provide both a way to effusively emulate RISC-V assembly and also provide a simple way of producing a graphical user interface. We identified three possible languages to implement the project in: Java \cite{sunmicrosystems_2022_java}, Haskell \cite{marlow_2010_haskell} and JavaScript \cite{ecmainternational_2023_ecmascript}.

A further consideration may of been to review lower-level languages as an implementation option, these would of provided significantly faster emulation in languages such as C and C++, which also run natively on windows devices, however these require more additional setup to work on other devices. Low level languages like C and C++ are more optimal for the emulation, but due to a lack of experience creating visual elements within each, the overhead of having to learn a low level language sufficiently alongside the project isn't feasible.

Of the languages discussed below, Java was selected as the ideal approach with a robust hierarchy system to allow for reusable and maintainable code, and a large quantity of help and plugins available online to speed up development, as well as a prior heavy use of Java providing a strong familiarity with the language. With Haskell and JavaScript not being chosen due to limited \ac{UI} options in Haskell with a higher learning curve, and a lack of rigid typing and superficial class handling pushing JavaScript out of the picture as discussed below.

\subsection{Java}
Java \cite{sunmicrosystems_2022_java} is a high-level object oriented programming language. It is designed to be written once and run anywhere, with Java code compiled into bytecode (an intermediary platform independent representation of code that can be executed on any machine using the Java Virtual Machine), that can then be run by a Java Virtual Machine on any device with an implementation. Due to this nature, Java is a excellent it for the project, with us being able to write an application that can be run on any device without having to explicitly code device dependent code.

Being object orientated, also provides numerous benefits, with us able to structure instructions as individual objects, making use of Java's inheritance system to extend base classes to group functionality and organise code in a logical way.

Java also supports the inclusion of external plugins, via directly importing, or build tools such as Maven \cite{porter_2022_maven} or Gradle \cite{gradleinc_2023_gradle} which provide a organised way to manage and download dependencies for projects, as well as versioning dependencies and application packages. With this we can make use of libraries to speedup development, avoiding re-implementing the wheel when far more efficient and robust options are available.

Thanks to Java's object oriented design, plugin support and being my preferred language of choice, it was chosen to be used for the project as it provides a robust way to implement RISC-V components is a logical way, with the option to make use of libraries such as JavaFX \cite{sunmicrosystems_2022_javafx}, Swing \cite{oconner_2007_using} or QtJambi \cite{omixvisualization_2023_omixvisualizationqtjambi} to produce familiar and intuitive user interfaces.

\subsubsection{JavaFX}
JavaFX \cite{sunmicrosystems_2022_javafx} is a \ac{UI} library produced originally by Chris Oliver in 2007. It provides methods to build GUI's that show natively on all platforms, without the end developer having to provide any device specific code. It also provides a simple HTML like templating language called FXML, that can be edited visually with a application called SceneBuilder \cite{gluon_2022_scene} which is a \ac{WYSIWYG} editor providing a drag and drop interface to build a responsive \ac{UI}.

\subsubsection{Swing}
Swing \cite{oconner_2007_using} is an older \ac{UI} library produced by Oracle. It was superseded by JavaFX but is still a viable option for \ac{UI} creation, with many older examples and templates available online. However, unlike JavaFX it supports no form of templating language requiring the entire \ac{UI} to be programmed. As a result it would of taken far longer to produce a \ac{UI} using Swing, and thus it was not chosen.

\subsubsection{QtJambi}
QtJambi \cite{omixvisualization_2023_omixvisualizationqtjambi} provides a way for Java to make use of the native Qt library written in C/C++, which is very powerful allowing for the creation of complex applications, with lot of common GUI components and a simple API. However, much like with Swing, it provides no templating language either, whilst being more complex than JavaFx. Thus, QtJambi whilst being an excellent framework and higly considered, was not chosen in favour of JavaFX's templating system and provided editor.

\subsection{Haskell}
Haskell \cite{marlow_2010_haskell} was also identified as a language that may be applicable to the project. Haskell is a functional language. It also allows you to create new types and methods to operate on them. Haskell would of proved very useful for parsing a user program into a intermediate language that could then be emulated nicely in Haskell, with error parsing wrapped nicely inside.

Much like Java, Haskell provides a way to include packages (Cabal \cite{haskell_2023_the}) created by others to avoid rewriting complex code, and simplify creating user interfaces. Unlike Java, Haskell code must be compiled to machine code for every intended target and the distributed, instead of compiling to byte-code and then being run by a virtual machine on each target device. This means having to distribute multiple applications instead of one.

Haskell was ultimately not chosen due to its poor range of GUI libraries, with many of them being old and outdated, or limited in what they provide such as Gtk2Hs \cite{haskell_2023_gtk2hs} CITE or Threepenny-gui \cite{threepennygui_2017_about} CITE . Further, with a limited knowledge of Haskell and insufficient time to learn Haskell whilst concurrently coding the project, it was decided to be unfeasable.

\subsection{JavaScript and TypeScript}
JavaScript was the final language identified for the project. JavaScript would of made the project web-based, or partly web-based. JavaScript is a web scripting language with a syntax based on Java and C following the ECMAScript Specification \cite{ecmainternational_2023_ecmascript}.

JavaScript would of provided a very simple way to create and manage the user interface, with many libraries being available to simplify \ac{UI} building such as React \cite{metaopensource_2023_react}, and animation libraries such as ThreeJS \cite{cabello_2023_threejs} to speedup the development process. However, directly implementing the emulation side in JavaScript would be much harder due to JavaScript not making rigid use of types and performing odd behaviour when attempting to guess type conversions.

This could be remedied by using Typescript \cite{microsoftcorporation_2020_javascript}, which forces types onto JavaScript whilst also providing features such as interfaces much like in Java. Whilst JavaScript provides classes and object orientated programming, they aren't the most rigid of structures. Thus, building a complex emulation system would require a large amount of type checking if not using TypeScript. Also, by being a web-based language it is limited to an individuals browser, with JavaScript being slower than native languages and possibly suffering with a more complex emulation. As a result neither JavaScript nor TypeScript were  chosen for the project, but would make a suitable extension, if a web interface were desired in the future.

\section{Existing Solutions}

It is important to consider the strengths and weaknesses of current solutions to simulating architectures. In this case, we have considered 4 existing solutions, of which 3 are based on RISC-V (Emulsiv \cite{savaton_2023_eseotechemulsiv}, Cornell Interpreter \cite{cornelluniversity_riscv} and rvemu \cite{doi_2021_d0iasmrvemu}), and another (LittleManComputer \cite{higginson_2014_little}) aimed at being far simpler for school use.

The project aim to combine the pro's of the below solutions whilst trying to avoid as many of the cons as possible, to produce a well rounded solution that is accessible to all, whilst providing a deeper level of understanding without becoming cluttered or limited to specific instruction sets.

\subsection{LittleManComputer}\label{sec:lmc}
\begin{figure}[H]
    \centering
    \includegraphics[width=0.75\linewidth]{dissertation/DATA/LMC.jpg}
    \caption{LittleManComputer Simulator by Peter Higginson}
    \label{fig:lmc}
\end{figure}
LittleManComputer \cite{higginson_2014_little} is a instructional model of the computer created in 1965. In 2014, Peter Higginson created a visual simulator \cite{higginson_2014_little} for LMC.

Built purely in JavaScript, it provides a simple interface to visualise the interactions of LMC as instructions manipulate data flowing around the registers and memory. The overall visualisation it provides is fairly intuitive to follow, simple and to the point, with no unnecessary additions or complexity. As a result it accessible to a wider range of users with less of a learning curve, complete with a conclusive help documentation.

\begin{table}[h]
\begin{tabular}{|p{0.5\linewidth} | p{0.5\linewidth}|}\hline
\textbf{Pros}                                                     & \textbf{Cons}                             \\\hline
Simple to understand and follow                                   & Not based on RISC-V                       \\\hline
Applicable to all age ranges, younger user will have little issue & Too simplistic for our use                \\\hline
Supports text I/O                                                 & Limited register operations \\\hline
\end{tabular}
\end{table}

\subsection{Emulsiv}\label{sec:emulsiv}
\begin{figure}[H]
    \centering
    \includegraphics[width=0.85\linewidth]{dissertation/DATA/EMULSIV.jpg}
    \caption{Emulsiv by Guillaume Savaton}
    \label{fig:emulsiv}
\end{figure}
Emulsiv \cite{savaton_2023_eseotechemulsiv} is a web based RISC-V simulator. It was produced by Guillaume Savaton for ESEO (a French Electronic and Engineering School) to help students understand the inner workings of a RISC-V processor. 

Emulsiv provides a sophisticated visualisation of the processor whilst still being easy to follow. However, animations don't follow the drawn path between components, instead choosing to drift across the screen, whilst highlighting the respective path which may be confusing for some users. Emulsiv also provide a lot of detail on screen, which while useful, may also be a burden for those less familiar with RISC-V, posing as a deterrent from using the simulator.

\begin{table}[h]
\begin{tabular}{|p{0.5\linewidth} | p{0.5\linewidth}|}
\hline
\textbf{Pros}                                         & \textbf{Cons}                                                                         \\ \hline
Makes use of the complete base 32 bit instruction set & Provides no way to extend with additional instruction sets                            \\ \hline
Provides Text, General Purpose and Bitmap I/O         & Has no central control unit                                                           \\ \hline
Simple design and animations                          & Overall usage is somewhat complex for beginners, and requires thorough reading to use \\ \hline
\end{tabular}
\end{table}

\subsection{Cornell Interpreter}
\begin{figure}[H]
    \centering
    \includegraphics[width=0.75\linewidth]{dissertation/DATA/CORNELL.jpg}
    \caption{Cornell RISC-V Interpreter}
    \label{fig:cornell}
\end{figure}

The Cornell RISC-V Interpreter \cite{cornelluniversity_riscv} was made for the Cornell University CS3410 course. It provides a simple web interface to interpret RISC-V code showing a display of the register and memory changes only.

Unfortunately, it only supports a subset of the base 32 bit implementation which greatly limits its potential, however this decision does provide the benefit of constraining the system to a simpler form that is easier for those new to RISC-V to explore without having the option of many other instructions being present, instead focusing on the core load-store nature of RISC-V.

\begin{table}[h]
\begin{tabular}{|p{0.5\linewidth} | p{0.5\linewidth}|}
\hline
\textbf{Pros}                                         & \textbf{Cons}                                                                         \\ \hline
Makes use of the complete base 32 bit instruction set & Provides no way to extend with additional instruction sets                            \\ \hline
Provides Text, General Purpose and Bitmap I/O         & Has no central control unit                                                           \\ \hline
Simple design and animations                          & Overall usage is somewhat complex for beginners, and requires thorough reading to use \\ \hline
                                                      &  Very limited overall visualisation options                                                                                      \\ \hline
\end{tabular}
\end{table}

\subsection{rvemu}
\begin{figure}[H]
    \centering
    \includegraphics[width=0.85\linewidth]{dissertation/DATA/RVEMU.jpg}
    \caption{rvemu by Asami Doi}
    \label{fig:rvemu}
\end{figure}
rvemu \cite{doi_2021_d0iasmrvemu} is a RISC-V emulator created by Asami Doi in Rust. It provides a CLI interface to load and emulate written RISC-V code. It supports the full 64 bit implementation and many of the 64 bit extensions. However, it provides no visualisation whatsoever other than the end results of a scripts execution.

As a result of providing no visualisation rvemu is very efficient and can emulate large scale RISC-V programs with ease, which is ideal for cases where a prospective user only wishes to know the final result. By supporting many of the 64 bit extensions, it provides a capability to emulate significantly more advanced programs than other simulators focused more on visualisation aspects. However, more technical users may wish for a visualisation of these more complex extensions, which it provides a lack off, which is something to consider for the project in terms of how complex visualisation becomes and to what extend we visualise extensions.

\begin{table}[H]
\begin{tabular}{|p{0.5\linewidth} | p{0.5\linewidth}|}
\hline
\textbf{Pros}                                                        & \textbf{Cons}                                           \\ \hline
Simple to use                                                        & Doesn't support the full base 32 bit instruction set    \\ \hline
Simple and easy to follow updating of registers and memory           & No way to extend with additional instruction sets       \\ \hline
Allows stepping of instructions, and to simulate at different speeds & No visualisation other than register and memory changes \\ \hline
                                                                     & Very limited overall visualisation                      \\ \hline
\end{tabular}
\end{table}

\section{Nielson Usability Heuristics}\label{sec:nielson}
Our design should be easy to use and well structured, however a was to measure this is required, to ensure that what we think is acceptable actually is in the real world.

The Nielsen Usability Heuristics \cite{nielsen_2020_10} denote 10 principles for \ac{UI} design, and are treated as a rule of thumb and not specific guidelines. The 10 principles (as copied from the website \cite{nielsen_2020_10}) are:

\begin{enumerate}
    \item Visibility of system status - The design should always keep users informed about what is going on, through appropriate feedback within a reasonable amount of time.
    \item Match between system and the real world - The design should speak the users' language. Use words, phrases, and concepts familiar to the user, rather than internal jargon. Follow real-world conventions, making information appear in a natural and logical order.
    \item User control and freedom - Users often perform actions by mistake. They need a clearly marked "emergency exit" to leave the unwanted action without having to go through an extended process.
    \item Consistency and standards - Users should not have to wonder whether different words, situations, or actions mean the same thing. Follow platform and industry conventions.
    \item Error prevention - Good error messages are important, but the best designs carefully prevent problems from occurring in the first place. Either eliminate error-prone conditions, or check for them and present users with a confirmation option before they commit to the action.
    \item Recognition rather than recall - Minimise the user's memory load by making elements, actions, and options visible. The user should not have to remember information from one part of the interface to another. Information required to use the design (e.g. field labels or menu items) should be visible or easily retrievable when needed.
    \item Flexibility and efficiency of use - Shortcuts, hidden from novice users, may speed up the interaction for the expert user so that the design can cater to both inexperienced and experienced users. Allow users to tailor frequent actions.
    \item Aesthetic and minimalist design - Interfaces should not contain information that is irrelevant or rarely needed. Every extra unit of information in an interface competes with the relevant units of information and diminishes their relative visibility.
    \item Help users recognise, diagnose, and recover from errors - Error messages should be expressed in plain language (no error codes), precisely indicate the problem, and constructively suggest a solution.
    \item Help and documentation - It's best if the system doesn't need any additional explanation. However, it may be necessary to provide documentation to help users understand how to complete their tasks.
\end{enumerate}

Other principle concepts exist, often be a re-amalgamation of Nielsons, to focus on specific areas of \ac{UI} design, with most aiming for consistency as in principle 4, simplicity and focusing on the user by placing them at the centre of design \cite{uxpin_2020_the}.

\chapter{Requirements}
\label{ch:requirements}
The projects objectives are split into two parts: Emulation and Visualisation. This decision was made due to the emulation being the core of the system, and without it any form of visualisation would be useless, with the visualisation depending on specific emulation outputs.

Our requirements follow the MoSCoW method, in which each requirement is denoted as either \textbf{Must}, \textbf{Should}, \textbf{Could} or \textbf{Would}, allowing us to produce a prioritised list of requirements, focusing on the \textbf{Must's} first ensuring core parts of the project are implemented first, without straying to implement unnecessary additions that can be done towards the end.

\begin{enumerate}
    \item \textbf{(M)} Emulation of the base instruction set: RV32I \cite{waterman_2019_the} (The base 32 bit implementation of RISC-V), as per the RISC-V specification,
    \item \textbf{(M)} A lexical analysis of inputted RISC-V code to ensure proper syntax is used and inputted code can be successfully parsed,
    \item \textbf{(S)} A semantic analysis of inputted RISC-V code to ensure that lexically valid code also conforms the expected values of any given instruction,
    \item \textbf{(C)} Loop control via labelled functions
    \item \label{req:md} \textbf{(S)} Emulation of the "Standard Extension for Integer Multiplication and Division" extension,
    \item \label{req:fp} \textbf{(C)} Emulation of the "Standard Extension for Single-Precision Floating-Point" extension,
    \item \textbf{(M)} A visualisation of data moving across the processor using JavaFX \cite{sunmicrosystems_2022_javafx},
    \item \textbf{(C)} A visualisation of instructions being accessed and decoded using JavaFX \cite{sunmicrosystems_2022_javafx},
    \item \textbf{(M)} A comprehensive system allowing for the display of data moving around the processor including:
    \begin{enumerate}
        \item \textbf{(M)} Numerical data moving from memory to other components and vice-versa,
        \item \textbf{(S)} Addressing requests,
        \item \textbf{(C)} Manipulation of multiple values simultaneously to simulate the effect of processor operations (e.g. addition, subtraction, shifts, etc),
        \item \textbf{(M)} Control of animation speed
        \item \textbf{{C}} Instruction stepping
    \end{enumerate}
    \item \textbf{(M)} The code-base will be well documented with maintainable code
\end{enumerate}




\chapter{Design}
\label{ch:design}
The design of the project consist of three individual sections, that are then combined to produce a resulting application. These are: The Emulator \ref{sec:emulator}, Visualisation \ref{sec:visualisation} and the Module System \ref{sec:module}. The module system was not originally a section, but was considered later on and then designed to allow for RISC-V extensions to be included. The use of "extension" and "module" may be confused within this section, however it should be identified as: "We create a module that encapsulates the logic of a RISC-V extension".

This approach was chosen to allow for a consistent flow of development, with the Emulator being core to allow the visualisation system to know what to simulate, thus the emulator must be completed first, and then the visualisation built separately and then combined onto of the emulator. The module system can then be attached on later once it was designed. 

Together these 3 sections encapsulate the entire project and its complete functionality, and the inclusion of the Module system allows us to satisfy requirements \ref{req:md} and \ref{req:fp} to include the Multiply and Divide, and Single Precision Floating Point Module. And, with these 3 sections being created completely from scratch it provides the flexibility to independently design and create each, and then combine them later on when required, allowing for a detached design and implementation process.

\begin{figure}[h]
    \centering
    \includegraphics[width=0.95\linewidth]{dissertation/DATA/sys architecture.png}
    \caption{Overall System Architecture}
    \label{fig:sys_architecture}
\end{figure}

In the system architecture diagram in Figure \ref{fig:sys_architecture} we provide an outline of how the 3 sections integrate within the simulator. A user will directly interact with the visualisation layer, in turn the visualisation layer interacts directly with the emulator and module system to provide functionality. This approach was decided upon as the end user should never directly interact with the emulator as this avoids the simulation aspect, and the user should have limited access to the module system to enable, disable, load and unload modules as they require without having to worry about how this impacts the rest of the visualisation or emulator.

Within the system architecture diagram "Module Storage" is provided. This will exist on the users physical machine to store loaded modules to avoid loosing functionality over application restarts.

It is important to consider how a user may interact with the system in everyday use. The user sequence diagram in Figure \ref{fig:user_sequence} describes a typical user flow. The flow covers how a user may start by launching the application and entering a RISC-V program. By hitting "Run" the code is either accepted via a successful parse or rejected and a error is returned to the user. This may repeat several times, until a successful parse occurs. When this happens the program is executed line per line with animations playing out sequentially. After the user decides to manage their modules via the module popup, enabling and disabling the listed modules with a confirmation being shown after each change.

\begin{figure}[h]
    \centering
    \includegraphics[width=0.95\linewidth]{dissertation/DATA/sequence diagram.png}
    \caption{Example user sequence diagram}
    \label{fig:user_sequence}
\end{figure}

\section{Emulator}\label{sec:emulator}
The emulator is designed to mock the RISC-V architecture closely, whilst also being simple and efficient. In the overall system architecture diagram in figure \ref{fig:sys_architecture} the emulator is denoted as flowing between 3 high level ideas: "Parser", "Execution" and "Output". These are not rigid names, but allow us to comprise a high level overview of the entire emulator with each being discussed below:

\subsection{Parser}\label{sec:parser}
In order for any code to be emulated it must be parsed first. The design for the parser must be simple and efficient, providing suitable error feedback when required.

The parser is designed to consume the inputted user program line-by-line. Each line is expected to conform to a valid instruction with a set amount of operands. For example the following valid ADDI instruction will read register x1, add 3 to it and then write the result back to register x1:\\\\
\verb|ADDI x1, x1, 3|
\\\\
The parser should first identify the instruction name, and then identify the respective operands that the instruction should expect, including how many and of what type each. This can be performed by linearly consuming each inputted operand and checking it conforms to the expected type, whilst pre-checking that there aren't more or less operands than expected, rejecting if anything is invalid, throwing an exception if so. If an instruction is valid, it should be completely consumed and converted into an intermediary instruction format for use later.

\begin{figure}[H]
    \centering
    \includegraphics[width=0.95\linewidth]{dissertation/DATA/instr_abstract_uml.png}
    \caption{Basic UML diagram of the Abstract instruction class}
    \label{fig:instr_abstract_uml}
\end{figure}

Figure \ref{fig:instr_abstract_uml} presents this intermediary form. This abstract \texttt{Instruction} class will encapsulate each individual instructions logic, with every instruction extending and overriding the \verb|execute()| method. Figure \ref{fig:instr_abstract_uml} shows an example of an ADD instruction being implemented.

Each individual implementation should be stored centrally and referenced by name such that it may be called later for execution with its parsed operands stored separately such that each instruction may be reused with different operand combinations.


\subsection{Execution}
Execution encapsulates the internal design of the system including the Registers, Memory and how they integrate. There needs to be a consideration first on how both will store a 32 bit binary value. This could simply be a string of of 32$\times$1's and 0's or the based integer type in Java. However the use of a custom \texttt{Binary} object \ref{fig:bin_regmem_uml} would seem more logical to provide consistency between the Register and Memory, with simple functions to both read and write as Binary, Denary and Hex.

The emulation requires a representation of the 32 base registers, with the ability to reference them by not only their name. but also their respective Application Binary Interface (ABI) name as well \cite{riscvinternational_2014_calling}(page 3) which is an alternate name denoting specific usage or idea placement of data. 

A list might work nicely here, allowing direct addressing from 0-31 for the base 31 registers, however if a user addressed via their ABI name we would have to linearly search every register to check its ABI name which is inefficient. Thus a map suites the problem better, with the ability to directly map both the numerical name and ABI name to each register with the ability to read and write via these names as seen in Figure \ref{fig:bin_regmem_uml}, and providing a O(1) lookup time as a additional benefit.

Unlike registers, memory needs to be addressable by its 4 byte aligned location, and thus we could opt for two methods of referencing:
\begin{enumerate}
    \item Use a list and take modulo 4 of addressed to get the relative index in the list for each memory value.
    \item Use a map to store the location as a key, with the memory value as the value, avoiding extra computation to calculate locations.
\end{enumerate}
Both are valid options. However, option 2 provides the benefit of a O(1) lookup time, compared to O(n) for a list, and seeing as the memory may become infinitely large with a complex emulation, option 2 suffices as the better choice.

\begin{figure}[H]
    \centering
    \includegraphics[width=\linewidth]{dissertation/DATA/bin_regmem_uml.png}
    \caption{Binary, Memory and Register UML Diagram}
    \label{fig:bin_regmem_uml}
\end{figure}

\subsection{Output}
Our output design simply come in the form of the handshake it will have with the visualisation section. The Emulator itself should only output the end result, parse errors and signals for animations to take place.

Each should be simple to differentiate, most likely with parse errors being raised as an error by Java that can be caught and passed to the visualisation to be displayed. The end output being printed to the terminal during development and for debugging, and animation signals being preemptively stubbed in code, to be hooked into later by the visualisation system.






\section{Visualisation}\label{sec:visualisation}
The visualisation of the emulation is a large part of the project, with the overall simulation relying heavily on a well designed \ac{UI}, thus the design of the \ac{UI} has been heavily considered to ensure an appropriate interface that presents the right amount of information without being too simple nor too complicated.

\begin{figure}[H]
    \centering
    \includegraphics[width=\linewidth]{dissertation/DATA/early_design.jpg}
    \caption{Original hand drawn UI design}
    \label{fig:early_ui_design}
\end{figure}

Our first \ac{UI} design (Figure \ref{fig:early_ui_design}) was designed to be simple, and loosely based on the design of LittleManComputer (Figure \ref{fig:lmc}) and Emulsiv (Figure \ref{fig:emulsiv}) as discussed in Section \ref{sec:lmc} and \ref{sec:emulsiv} respectively. 
The design places the code editor on the left, the main visualisation area in the centre, registers on the right, memory along the bottom and a menu bar stretching the top. These positions were specifically chosen based on the constraint of element sizing:
\begin{itemize}
    \item The code editor requires a large vertical space to hold many lines of code, but a relatively short horizontal width, due to the average instruction being relatively short.
    \item The animation/visualisation area was given the most space as this is the main focus point of the application, and thus being dead centre is the most appropriate location within the \ac{UI}.
    \item Registers, much like the code editor require a small horizontal space to display their value, but 32 of them need to be on display at any given time, thus a large horizontal space allows for this with minimal scrolling on smaller screens.
    \item Memory also requires limited horizontal space, however, unlike registers, it starts of empty for every execution, and may not be used in any given execution, thus designating it to the bottom to fill waste space is more appropriate. Should the memory fill up, the user may scroll through it to see all the values at any given time.
\end{itemize}

Other elements have also been positioned for their own respective reasons irrespective of size constraints:
\begin{itemize}
    \item The menu bar stretches along the top to conform with expected user patterns and expectations, with users commonly expecting it at the top of the application, thus placing it elsewhere would of been unprofessional.
    \item The Settings and Module sections are popups, that display in the middle of the screen. This conforms with user expectations and grabs attention to the popup without having to search for it on screen. It also provides the option to be moved based on the users need, whilst hiding secondary content that is important to the core simulation.
\end{itemize}

As-well as the overall \ac{UI} design, a simple yet intuitive layout needed to be created to visualise the internal data movement around the processor components (\ac{ALU}, \ac{CU}, \ac{IR}, \ac{ID}, Instruction Memory, Registers and Memory). The original design of this can be seen below in Figure \ref{fig:early_animation_design} with, Control Unit designs in Figure \ref{fig:early_cu_design}.

\begin{figure}[H]
    \centering
    \includegraphics[width=\linewidth]{dissertation/DATA/animation_layout.jpg}
    \caption{Original animation area design}
    \label{fig:early_animation_design}
\end{figure}

\begin{figure}[H]
    \centering
    \includegraphics[width=\linewidth]{dissertation/DATA/control_unit.jpg}
    \caption{Original Control Unit design}
    \label{fig:early_cu_design}
\end{figure}

Figure \ref{fig:early_animation_design} displays the concept of the animation area, with components being connected by coloured wires, with blue, red and black representing the transfer of Instructions, Data and Control signals respectively. Allocating larger boxes to side elements that will correspond to the respective \ac{UI} elements on the wider interface.

The \ac{CU} in Figure \ref{fig:early_animation_design} is just displayed as a simple box, however it is intended that it will display mroe data to convey what the emulator is doing, for example what stage is currently operating and some information associated with the stage. Examples of this more detailed design can be seen in Figure \ref{fig:early_cu_design}.

These base designs allowed us to then revise and redesign to improve aspects. During revisions we decided to consider Nielsen's Usability Heuristics \cite{nielsen_2020_10} as covered in section \ref{sec:nielson}. Within this, our design failed 2 principles being 1 and 8 relating to visibility of system  status and aesthetic and minimalist design.

The design failed these due to our choice to redesign to include colour within our interface to help differentiate the animation components. This lead to the used of colours such as red and green which colourblind users are unable to differentiate between, instead appearing as a murky brown. To prevent this issue whilst also maintaining this newer design we switched to using grayscale. This eliminated the issue whilst still providing a way for colourblind users to easily differentiate between processor components and thus making the visability of the system better available to all.

A further design issue was a choice of background colour, which a original change to use a gray background instead of white, going with a darkish gray. However, this produced a low contrast ratio between the background and other element which may prove hard for visually impaired users to see as-well as increasing eye strain. The Web Accessibility Initiative \cite{webaccessibilityinitiativew3_2022_understanding} denotes a minimum contrast ration of 4.5:1, which our original design of gray on gray doesn't provide. Upon further redesigning we switched this dark gray for a much lighter gray, which in turn greatly increased the contrast ratio to an acceptable level.

An overview of the above changes and a few extra are listed below:
\begin{enumerate}
    \item Lightening of the background to increase the contrast ration to an acceptable level,
    \item Convert the red, green and blue boxes to grayscale to alleviate the issue for colourblind users,
    \item Gray boxes were switched to purple to increase contrast.
\end{enumerate}

With all these consideration taken into account the final design was complete, with physical mock up in Figure \ref{fig:final_implemented_design}. (Please note this image has been taken from our implemented design due to the designed version being lost)

\begin{figure}[H]
    \centering
    \includegraphics[width=0.9\linewidth]{dissertation/DATA/final_design.jpg}
    \caption{Final full UI design}
    \label{fig:final_implemented_design}
\end{figure}

This final design allows us to follows Nielsen's principles \cite{nielsen_2020_10}, with a simple and minimalist design, with only whats required on screen, and further principle 4, keeping our layout and design consistent across the entire application.





\section{Module System}\label{sec:module}
The applications module system is designed to be convenient and intuitive to use. As mentioned at the start of the design section, modules are implementations of RISC-V extensions and were a later addition that was designed around the emulator. But, could be easily designed in thanks to our simple emulator design, with the ability to directly add new instruction into our map design and reference them via their respective names.

\begin{figure}[H]
    \centering
    \includegraphics[width=0.9\linewidth]{dissertation/DATA/module uml.png}
    \caption{Module System UML Diagram}
    \label{fig:module_uml}
\end{figure}

Modules are designed to implement a simple module interface (Figure \ref{fig:module_uml}). With the system designed to be able to find and load modules, and then enable and disable the mas required. Modules them self are designed to allow for quick creation to allow the implementation of custom logic, with an enable and disable method providing an API instance discussed below.

This API instance is designed to be simple to allow modules to access the core of the emulator to allow the addition of exra instructions and logic, whilst restricting access to internal systems that shouldn't be modified. 

The overarching system managing individual modules was also designed to be simple, with an option to quickly enable and disable modules, aswell as a simple interface to load and remove modules.

Figure \ref{fig:module_popup} shows an example of the module popup, which will allow users to manage there modules, aswell as enable and disable them.

\begin{figure}[H]
    \centering
    \includegraphics[width=0.7\linewidth]{dissertation/DATA/module design.jpg}
    \caption{Module popup design}
    \label{fig:module_popup}
\end{figure}



\definecolor{codegreen}{rgb}{0,0.6,0}
\lstdefinestyle{mystyle}{
    keywordstyle=\color{magenta},
    commentstyle=\color{codegreen},
    language=Java,
    basicstyle=\ttfamily\footnotesize,
    breakatwhitespace=true,         
    breaklines=true,                 
    captionpos=b,                    
    keepspaces=true,                 
    numbers=left,                    
    numbersep=5pt,                  
    showspaces=false,                
    showstringspaces=false,
    showtabs=false,                  
    tabsize=2
}
\lstset{style=mystyle}


\chapter{Implementation}
\label{ch:implementation}
The project has been implemented in 3 stages, similar to the 3 sections of our design. These are: The Emulator described in Chapter \ref{sec:impl_emul}, Visualisation in Chapter \ref{sec:impl_vis} and the Module System in Chapter \ref{sec:impl_mod}. 

The module system was a later addition to the project, hence its separate design and implementation sections, instead of being directly implemented from the beginning, its respective changes and alterations to the implementation are discussed in its respective section.

First a few bits of terminology should be covered:

\begin{itemize}
    \item \textbf{Instance} - An instance is a physical manifestation of a class during runtime. It can be physically interacted with, providing the written functionality.
    \item \textbf{Abstract Class} - An abstract class is a class that cannot be directly instantiated. It provides a set of implemented and unimplemented methods (labelled abstract). The abstract class must be extended, with any abstract methods implemented.
    \item \textbf{Interface} - An interface is an abstract type that declares behaviour a class must implement, containing a set of method and constant declarations that any implementing class must implement.
\end{itemize}

\section{Emulator}\label{sec:impl_emul}
At its base the emulator provides implementations of the core components to emulate RISC-V instructions. These are representations of physical register, memory and instructions. These three allow us to completely emulate any RISC-V instruction. A further part of the emulator is the parser, which provides essentially functionality to covert user written programs into a representation the emulator can understand.

Below, each of the 3 core components is discussed as well as the other relevant details allowing for the complete implementation of the emulator.

\subsection{Binary Class}
Originally the emulator stored the respective binary values of register and memory values as a integer, due to the fact that RISC-V deals with 32 it twos-compliment numbers, and Java's integer is also 32 bit twos-compliment. Whilst some instructions suited this making use of just basic arithmetic operations, it was quickly noted that instructions started to operate on the binary form.

It was decided to switch to a more intuitive representation as noted in our design in Figure \ref{fig:bin_regmem_uml}, by creating a custom \texttt{Binary} class. This class holds a string representing the 32 ones and zeros making up the value, which can then be retrieved as denary, hex or the binary value itself.

Further, additional methods could be provided directly on the \texttt{Binary} class, including the ability to sign extends values, get a certain amount of bits and also directly retrieve a new instance directly from an integer value as seen in Listing \ref{lst:binary}.

\begin{lstlisting}[caption=Additional Binary Methods, label=lst:binary]
public class Binary {
    ...
    public Binary getBits(int bits) {
        return new Binary(this.value.substring(this.value.length()-bits, this.value.length()));
    }

    public Binary getBitsSignExtended(int bits){
        Binary b = new Binary("0");
        b.setSignExtendedValue(this.value.substring(this.value.length()-bits, this.value.length()));
        return b;
    }

    public static Binary of(int value) {
        return new Binary(Integer.toBinaryString(value));
    }
}
\end{lstlisting}

As a result of switching from storing value as a integer to using this \texttt{Binary} class, it streamlined our implementation and provided a consistent way to manage binary values between the registers, memory and instructions whilst avoiding large amounts of conversion to and from integers.

\subsection{Registers}
Being the most core element of RISC-V, registers we the first component to be implemented. As per our design, each Register is represented as a instance of the \texttt{Register} class. As seen in Listing \ref{lst:register}, each register stores its name and alternate application binary interface name under the \texttt{name} and \texttt{nick} fields respectively, as well as a reference to its \texttt{Binary} instance storing the registers actual value.

\begin{lstlisting}[caption=Register Implementation, label=lst:register]
public class Register {
    private final String name, nick;

    private Binary binary;

    public Register(String name, String nick) {
        this.name = name;
        this.nick = nick;

        this.binary = new Binary("0");
    }

    // EXCLUDING ALL GET/SET METHODS 
    ...

    public void reset() {
        this.setValue("0");
    }
    public Register clone(){
        Register rnew = new Register(this.name, this.nick);
        rnew.setValue(this.getBinary());
        return rnew;
    }
}
\end{lstlisting}

The class also provides a host of methods to get the registers value, either directly returning the \texttt{Binary} instance via \texttt{getBinary} or returning representations of the value via the respective \texttt{getValueAs...()} method, which in turn calls the respective method on the \texttt{Binary} instance.

To represent the 32 base registers a mapping of each registers name and nickname is stored to the individual register. In order to conveniently store and access all these register a manager class called \texttt{RegisterSet} is used. An instance can be created specifying how many registers, a prefix, and a list of alternate names for each register, allowing it to be re-used for more than just the 32 base registers. This functionality cannot reside in the \texttt{Register} class itself since it would all have to exist in a static context which would violate object oriented programming principles \cite{wegner_1990_concepts}. It also avoids confusing the usage of \texttt{Register}'s as actual registers and also a management interface for retrieving and storing different registers concurrently.

For the specified amount of registers, a new instance of the \texttt{Register} class is created, and is added to a map data structure directly linking each register to its respective name, and optionally its alternate application binary interface name. This can be seen in Listing \ref{lst:reg_set}.

\begin{lstlisting}[caption=RegisterSet generation, label=lst:reg_set]
 public RegisterSet(String registerPrefix, int amount, List<String> registerNicks) {
    this.prefix = registerPrefix;
    this.registerMap = new HashMap<>();

    for (int i = 0; i < amount; i++) {
        String name = registerPrefix + i;
        String nick = registerNicks != null ? registerNicks.get(i) : null;
        Register register = new Register(registerPrefix + i, nick == null ? "" : nick);
        registerMap.put(name, register);
        if (nick != null) registerMap.put(nick, register);
    }

    RegisterSet.all.add(this);
}
\end{lstlisting}

This newly created instance of \texttt{RegisterSet} allows us to now interface with each register when needed. A individual register can be requested by specifying its name. The \texttt{RegisterSet} provides methods to directly read and write the integer or binary representation as a string/hex/integer value. Alternatively, a \texttt{Register} can be directly loaded and its instance used.

Within the write method, overloading is used to allow the same method to be called with different parameter types, which internally covert the data type and call the main write method, which locates and updated the respective register. Overloading works by allowing multiple methods share the same name, but expect different parameter types. Listing \ref{lst:write_overloading} demonstrating the write implementation with a base \texttt{write} method expecting a register name and binary string, and then two overloading methods taking in an Integer and Hex string respectively, converting it to a binary string and passing to our base method.

\begin{lstlisting}[caption=Write overloading, label=lst:write_overloading]
public void writeHex(String registerName, String hexValue) throws RegisterNotFoundException, InvalidValueException {
    int intHex = Utils.formattedHexToInt(hexValue);
    this.write(registerName, Integer.toBinaryString(intHex));
}

public void writeInt(String registerName, int intValue) throws RegisterNotFoundException, InvalidValueException {
    this.write(registerName, Integer.toBinaryString(intValue));
}

public void write(String registerName, String value) throws RegisterNotFoundException, InvalidValueException {
    if (ignoreZeroRegister(registerName)) return;

    // Check the given value is infact binary
    Utils.isBinary(value); // throws its own error

    if (!SettingsController.areAnimationsEnabled()) {
        Register reg = this.load(registerName);
        reg.setValue(value);
        return;
    }

    Register cloned = this.load(registerName).clone();
    cloned.setValue(value);
    this.toUpdate.add(cloned);
}
\end{lstlisting}

It is important to note that a new \texttt{Register} instance is created here and called to be updated. This originally wasn't the case and was directly updated in the map. However this change was made to improve visualisation performance, and it will be discussed further when exploring how the emulation and visualisation is integrated as seen in Section \ref{sec:impl_integ}.

A specific case is checked for registers. As per the RISC-V specification \cite{riscv_2015_riscv}, register \verb|x0| should not be writable, and remain a constant 0. Thus, we make a check for a write to this register, ignoring any that occur. This could pose an issue due to our reusable nature of the \texttt{RegisterSet} class. However, all of the extensions for RISC-V call for a constant 0 register, and if a new one should stray from this format, it is feasible to implement a check for if writing to 0 should be allowed or not.

\subsection{Memory}
\texttt{Memory} operates similarly to registers. However, memory is created on-demand as to not waste \ac{UI} and physical memory space as often values remain empty. The memory is also 4 byte aligned, rather than specifically addressed via apre-assigned name.

Each value in memory exists as an instance of the \texttt{MemoryValue} class. This class is a near carbon copy of the \texttt{Register} class in Listing \ref{lst:register}, with references to names removed, in turn of a reference to its location in memory. It may of been ideal to have both implement an interface representing the common methods. However due to their completely independent uses, this use of inheritance would of provided little functional benefit.

\texttt{MemoryValue}'s are stored in a map, with the map mapping integer locations to each \texttt{MemoryValue} inside a \texttt{Memory} instance. Unlike the \texttt{RegisterSet} class, the \texttt{Memory} class is not reusable. This choice was made for simplicity, and following the design of RISC-V, in which only one memory exists for data. 

As states earlier, each \texttt{MemoryValue} is only created when required. This is achieved through a \texttt{load(...)} method in Listing \ref{lst:mem_load} which makes use of the maps \texttt{computeIfAbsent} method. This returns the respective \texttt{MemoryValue} for the given location. Otherwise, a new instance is created and stored against the requested location in the map and also returned avoiding an additional \texttt{get()} call. This way allows us to avoid having to perform a check to see if the location has a value each time, and then create a new instance ourselves, leaving this functionality to the builtin method.

\begin{lstlisting}[caption=Memory load method, label=lst:mem_load]
public MemoryValue load(MemoryAddress address) throws InvalidValueException, RegisterNotFoundException, InvalidMemoryAddressException {
    return this.memoryMap.computeIfAbsent(address.getAddress(), x -> {
        MemoryValue mv = null;
        try {
            mv = new MemoryValue("00000000000000000000000000000000", address.getAddress());
        } catch (Exception e) {
            //
        }
        if (newValueBind != null) newValueBind.execute(mv); return mv;});
}
\end{lstlisting}

In the \texttt{load} method we chose to do nothing with an exception raised from an invalid address as a result of \texttt{address.getAddress()} as all calls to load should be after a call to the \texttt{checkAddress} method. This method firstly checks if the address is an the valid format, returning the integer address or throwing an \texttt{InvalidMemoryAddressException}, then checking it is aligned by taking modulo 4 of the address, throwing a \texttt{UnalignedMemoryAccessException} if not aligned, with the method returning nothing for a successful address, or throwing one of the exceptions for an invalid address.

The use of this slightly sophisticated memory address checking is required due to RISC-V's specification of memory addressing in the form of \verb|offset(register)| in which the given offset may be either a denary value, or a hexadecimal value. Each of which must specifically parsed, and checked, along with the actual register being checked to ensure it exists, as this format is not specific to the base 32 bit implementation. 

The exact implementation of the memory address parsing will be discussed further in the Parser section (\ref{sec:impl_parser}).

\subsection{Singleton vs Dependency Injection}
Within our register and memory implementation, both must be easily accessible to the rest of the program, especially instructions. There are two methods to achieving this: Singleton and Dependency Injection.

\subsubsection{Singleton}
The Singleton pattern is a \textbf{Creational Pattern} allows us to ensure that only a singular instance of a class exists. It provides global access to this instance, creating it on the first access, storing it and returning this same stored instance every time. Listing \ref{lst:singleton} shows an example of this as used with the \texttt{Memory} implementation, which can be obtained using \verb|Memeory.getMemory()|.

\begin{lstlisting}[caption=Singleton pattern using within the \texttt{Memory} class, label=lst:singleton]
private static Memory memory;

public static Memory getMemory() {
    if (memory == null) {
        memory = new Memory();
    }
    return memory;
}
\end{lstlisting}

This pattern however does provide an issue. It promotes the use of static allowing easy access to the instance without having to pass around a reference to the class. Unfortunately, for novice developers this often leads to the abuse of static to make variables accessible globally for ease of use, with this commonly being referred to as "static abuse" \cite{zivkovic_2021_3}, resulting in a deviation away from object oriented programming.

However, this pattern was chosen based on reducing deeply nesting dependency injection and providing a singular instance of Registers and Memory, making it difficult to accidentally create another.

\subsubsection{Dependency Injection}
On the other hand, Dependency Injection is a \textbf{Design Pattern}. Working via the method of passing a given instance through function parameters such it may then be used. Dependency Injection doesn't determine how many instances may exists, but still provides the ability to make any instance available when needed.

An example of Dependency Injection can be seen when the optional nickname list is passed into the \texttt{RegisterSet} class in Listing
\ref{lst:reg_set}. In this case an instance of a List is provided.

Dependency Injection is useful for passing around any instance of a class, but can become convoluted when a instance is required far down a function tree, which results in parameter drilling, with many methods having the provide the instance as a parameter, to pass it down to the method that needs it, which is often resolved by refactoring large chunks of code, or by switching to a singleton pattern when possible.

\subsection{Instructions}
The implementation of the instructions follows the design as outlined in Section \ref{sec:parser}, with an abstract \texttt{Instruction} class containing the base logic for every instruction. 

The \texttt{Instruction} class holds a reference to the registers and memory, as well as providing an \texttt{execute()} method that is abstract. Such that each extending class must implement it, allowing each implementation to provide its respective instruction logic. The \texttt{Instruction} class also specifies for a list of parsible types to be provided, which correspond to the respective instruction format and its expected inputs which will be discussed further in Section \ref{sec:impl_parser}.

\begin{lstlisting}[caption=Implemented ADDI Insstruction, label=lst:addi]
public class ADDI extends Instruction {

    public ADDI() {
        // Parameter types are specifeid in the constructure a EParseElement types
        super("ADDI", EParseElement.REGISTER, EParseElement.REGISTER, EParseElement.IMMEDIATE);
    }

    @Override
    public Animator execute(InstructionOperands operands) throws RegisterNotFoundException, InvalidValueException {

        int immediate = Utils.unknownStringIntegerToInt(operands.getR2(), 12); // Gets our 12 bit twos-compliment immediate (-2048 -> 2047)

        Register r = this.registers.load(operands.getR1());
        int result = r.getValueAsInteger() + immediate; // Read rs1 as its integer value and add it to the immediate
        this.registers.writeInt(operands.getDest(), result); // Store new value in destination register (rd)
        
        programCounter.increment();
    }
}
\end{lstlisting}

Listing \ref{lst:addi} is an example of the implemented \texttt{ADDI} instruction, specifying its name and operand types, and then overriding the \verb|execute()| method for its specific operation of adding an immediate value to the specified register R1, and writing it in the respective destination register.

In order to use each instruction, they need to be managed and tracked within the emulator. This is done via a managment class called the \texttt{InstructionManager}. Internally it stores a map dta structure linking each instructions name to its implemented class, with a method provided to return the instruction with a given name if it exists.

Before individual instructions can be used, each one must be registered to the \texttt{InstructionManager}. The naive approach would be to individually create an instance of each an add it to the map. However this would require writing \verb|new InstructionNameClass()| over 30 times. Instead we make use of the fact that Java allows us to instantiate classes programatically during run time. This is called reflection, in which code is able to inspect and modify other code.

Within this we are able to dynamically load each class extending the abstract \texttt{Instruction} class. Instead of writing the complicated logic to locate and load each instance, we make use of the Reflections Library \cite{ronmamo_2022_ronmamoreflections}. We specify that we would like to load any class that is a sub type of \texttt{Instruction}, and then iterate over each, getting its constructor and creating a new instance. The library then adds this new instance to the instruction map using the individual instructions stored name. Listing \ref{lst:instr_load} demonstrates this ability as implemented in the emulator.

\begin{lstlisting}[caption=Instruction loading using Reflections \cite{ronmamo_2022_ronmamoreflections} durign runtime, label=lst:instr_load]
public boolean findAndRegister() {

    if (!instructionMap.isEmpty()) return true;

    Reflections reflections = new Reflections("uk.co.noahdhollowell");
    Set<Class<? extends  Instruction>> instructions = reflections.getSubTypesOf(Instruction.class);

    instructions.forEach(instruction -> {
        try {
            Instruction i = null;
            i = instruction.getDeclaredConstructor().newInstance();

            if (instructionMap.containsKey(i.getIdentifier())) {
                return;
            }

            instructionMap.put(i.getIdentifier(), i);

        } catch (ReflectiveOperationException e) {}
    });

    return instructionMap.size() == instructions.size();
}
\end{lstlisting}

\subsection{Parser}\label{sec:impl_parser}
In order to make use of our \texttt{Instruction}'s, \texttt{Register}'s and \texttt{Memory}, user entered code must be parsed so that we can buildup a virtual list of instructions that can be emulated one by one.

\begin{lstlisting}[caption=Starting parse method, label=lst:parse_start]
 public Program parse(String programString) throws ParseException {
    this.program = new Program();
    this.currentLine = 0;

    if (programString.isBlank() || programString.isEmpty()) return null;

    String lines[] = programString.split("\\r?\\n");

    for (String line : lines){
        parseLine(line);
        currentLine++;
    }

    return this.program;
}
\end{lstlisting}

Parsing starts by taking user input and splitting it line by line as in Listing \ref{lst:parse_start}, then calling for each individual line to be parsed into its respective instruction call.

Before an line can be parsed, a intermediate format must be established to hold a parsed instruction and its operands. The \texttt{ExecutableInstruction} class provides this. The \texttt{ExecutableInstruction} is a wrapper class that holds a reference to the respective instruction and its operands to be used during execution. In a similar fashion to the \texttt{Instruction} class it provides a \verb|execute()| method that is called to perform the actual physical execution of the Instruction, directly calling the referenced instructions execute method with the given operands as seen in Listing \ref{lst:ei_execute}. 

This operation was chosen over directly calling the instruction as it simplifies the storing of instructions to be executed after parsing has complete. Without the wrapper both the instruction and operands to be run would have to be stored and fetched separately, and then combined. By using the \texttt{ExecutableInstruction} wrapper this double fetching and storing is eliminated with the wrapper automatically storing a reference to both and providing he combining to run the emulated instruction.

\begin{lstlisting}[caption=ExecutableInstruction class, label=lst:ei_execute]
public class ExecutableInstruction {

    private final Instruction instruction;
    private final InstructionOperands operands;

    public ExecutableInstruction(Instruction instruction, InstructionOperands operands) {
        this.instruction = instruction;
        this.operands = operands;
    }

    public void execute(Program parent, boolean runAnimations) throws InvalidValueException, RegisterNotFoundException, InvalidMemoryAddressException, UnalignedMemoryAccessException {
        this.instruction.execute(this.operands);
    }
}
\end{lstlisting}

Within this, instruction operands are stored within a \texttt{InstructionOperands} class storing a string array of all the parsed operands. With instructions mostly following the format of \verb|DEST R1 R2|. Referencing the destination register and then register 1 and 2 respectively, with the \verb|getDest() getR1() getR2()|, as well as \verb|getR(int i)| being available for retrieving these operands.

The decision to contain operands within a class with methods to access them was chosen over directly storing them in a string array to avoid confusion when writing individual instructions, and to provide a cleaner implementation.

With the \texttt{ExecutableInstruction} and \texttt{InstructionOperands} classes, line by line parsing can begin. Each line starts by being split on any spaces, with a limit of one split permitted such that only two parts are returned:

\begin{lstlisting}
String[] components = line.split(" ", 2); // Split on space, limiting two produce 2 parts
\end{lstlisting}

The first part of the split should be a valid instruction name. To check this we simply attempt to get the instruction by calling the \texttt{InstructionManager}'s \verb|getInstruction()| method with the parsed name. If the instruction doesn't exist then it will throw a \texttt{UnknownInstructionException}. This will then be caught and converted into a parse exception that is then thrown and returned to the user.

Next, the operands are parsed. These line in the 2nd index of the \verb|components| array. They are comma separated, and thus are split on each comma to get each individual operand, also removing any optional formatting white-space:
\begin{lstlisting}
String[] ops = components[1].replaceAll(" ", "").split(",");
\end{lstlisting}

Each operand is then individually parsed. This parsing is performed by making use of each instructions predefined operand types, which are stored as Enums:

\begin{lstlisting}
public enum EParseElement {
    REGISTER, MEMORY, IMMEDIATE;
}
\end{lstlisting}

Each of which are specified in the \texttt{Instruction} class constructor as seeing in the \texttt{ADDI} example in listing \ref{lst:addi}, and then retrieved during this part of parsing based on the previously successfully parsed instruction:
\begin{lstlisting}
instruction.getExpectedOps()
\end{lstlisting}

Each respective operand type has a respective parsing class based on a generic abstract class called \texttt{ParseElement} as seen in Listing \ref{lst:parse_element}. With the use of generics to allow for more advanced parsing in the future returning a type other than strings.

\begin{lstlisting}[caption=ParseElement generic abstract class, label=lst:parse_element]
public abstract class ParseElement<E> {
    protected final String value;
    protected final int line;

    public ParseElement(String value, int line) {
        this.value = value;
        this.line = line;
    }

    public abstract E parse() throws ParseException;
}
\end{lstlisting}

For parsing an immediate we simply attempt to convert the immediate to a integer. This is performed by a few utility functions which first assess if the string is in hexadecimal by checking if it starts with "x0", then converting the hexadecimal value to an integer. On the other hand for a denary it simply attempts to cast the denary to an integer, returning it if either were valid. Both make use of Java's builtin \verb|Integer.parsetInt(valie, base)| which allows is to specify the base of the value to parse, in this case being 16 for hexadecimal and 10 for denary.

For parsing a register we make use of Regex to identify a correct register. We adopt the following pattern: \verb|[a-z][0-9]{1,2}| which permits any lowercase letter once, and then between 1 and 2 digits for the register number. This allows us to parse register names efficiently. However, alternate names also need to be considered, which can be dealt with by simply querying if a register exists within the register set with the given name as seen in Listing \ref{lst:parse_reg}.

\begin{lstlisting}[caption=Register Parsing class, label=lst:parse_reg]
public class ParseRegister extends ParseElement<String> {

    public ParseRegister(String value, int line) {
        super(value, line);
    }

    @Override
    public String parse() throws ParseException {

        if (RegisterSet.getRegisters().getRegisterMap().keySet().contains(this.value)) return this.value;

        Pattern p = Pattern.compile("[a-z][0-9]{1,2}");

        Matcher m = p.matcher(this.value);

        if (!m.find()) throw new ParseException("The register '" + this.value + "' isn't valid!", this.line);

        return this.value;
    }
}
\end{lstlisting}

Finally to parse Memory addresses we can take the regex approach again. The offset for memory addressed may be either a denary value or a hexadecimal value, with each requiring a slightly different regex to parse. To facilitate this, memory addresses are stored as abstract class called \texttt{MemoryAddress} which provides a \verb|getAddress()| method to be implemented. Two sub classes extending this called \texttt{DenaryAddress} and \texttt{HexAddress} exist, each implementing the \verb|getAddress()| method.

Each uses a similar regex to parse the \verb|offset(register)| format:
\begin{itemize}
    \item Denary - \verb|((-?)[0-9]+)\(([a-z]([a-z]*[0-9]*)+)\)|
    \\ e.g. \verb|10(x1)|
    \item Hexadecimal - \verb|((-?)0x([0-9]*[a-f]*)+)\(([a-z]([a-z]*[0-9]*)+)\)|
    \\ e.g. \verb|0x1a(x23)|
\end{itemize}

With each then extracting the valid offset and register, loading the registers value and applying the offset as seen in Listing \ref{lst:hex_address}, returning a valid memory address or an exception if the parse fails.

\begin{lstlisting}[caption=Hex memory address parsing, label=lst:hex_address]
public class HexAddress extends MemoryAddress {
    public HexAddress(String address) {
        super(address);
    }

    @Override
    public int getAddress() throws InvalidMemoryAddressException, RegisterNotFoundException, InvalidValueException {

        Pattern p = Pattern.compile("((-?)0x([0-9]*[a-f]*)+)\\(([a-z]([a-z]*[0-9]*)+)\\)");

        Matcher m = p.matcher(this.address);

        if (!m.find()) throw new InvalidMemoryAddressException(this.address);

        String offset = m.group(1);
        String register = m.group(4);

        RegisterSet i32 = RegisterSet.getRegisters();

        int registerValue = i32.load(register).getValueAsInteger();


        return Utils.formattedHexToInt(offset) + registerValue;
    }
}
\end{lstlisting}

With these three parsing implementations for immediate's (hex/denary values), registers and memory, the operand parse loop (see Listing \ref{lst:op_parse_loop}) can now attempt to parse each operand one by one, calling all three individual operand parser's corresponding to the expected operand type. This results in either all the operands being successfully parsed and wrapped into a \texttt{InstructionOperands} instance, or a parse exception being thrown as a result of a syntax error, or an unknown input.

\begin{lstlisting}[caption=Operand parse loop, label=lst:op_parse_loop]
for (int i=0; i < ops.length; i++){
    //System.out.println(instruction.getExpectedOps()[i].toString()+": " + ops[i]);
    switch (instruction.getExpectedOps()[i]) {
        case REGISTER -> {
            new ParseRegister(ops[i], currentLine).parse();
        }
        case MEMORY -> {
            new ParseMemory(ops[i], currentLine).parse();
        }
        case IMMEDIATE -> {
            new ParseImmediate(ops[i], currentLine).parse();
        }
        default -> {
            throw new ParseException("Expected an input type of  '" + instruction.getExpectedOps()[i].toString() + "' but got '" + ops[i] + "' instead!", currentLine);
        }
    }
}

InstructionOperands io = new InstructionOperands(ops);
this.program.addInstruction(instruction, io);
\end{lstlisting}

Finally with a successfully parsed instruction and operands an instance of \texttt{ExecutableInstruction} is created and added to a list inside a \texttt{Program} class (discussed in the next section).

\subsection{Emulating}
With programs now parsed into the intermediary \texttt{ExecutableInstruction} instances we can begin to emulate functionality.

The \texttt{Program} class contains the previously mentioned list of \textbf{ExecutableInstruction} instances, as well as methods to start and stop the emulation of the given program. 

Emulation starts in the \texttt{Emulator} class. It initialises the registers and memory, whilst ensuring that instructions have been loaded into the emulator before emulating is allowed.

\begin{lstlisting}[caption=Execute method in the Emulator Class, label=lst:emulator]
public void emulate(String code) throws InvalidValueException, RegisterNotFoundException, InvalidMemoryAddressException, UnalignedMemoryAccessException, ParseException {
    ProgramParser pp = new ProgramParser();
    Program program = pp.parse(code);

    program.setRunAnimations(SettingsController.areAnimationsEnabled());

    if (program == null) return;
    this.currentProgram = program;

    program.execute();

    if (!SettingsController.areAnimationsEnabled()) {
        refreshAllRegisters();
        MainWindowController.getInstance().getMemoryController().refresh();
    }
}
\end{lstlisting}

Emulation is started by calling \verb|emulate| with user inputted code as seen in Listing \ref{lst:emulator}. This code is parsed into a \texttt{Program} which is then run by calling its \texttt{execute()} method. The program then sequentially calls each \texttt{ExecutableInstruction}'s \verb|execute()| method, until it reaches the end of the list. This is performed using callbacks, in which once the current instruction has finished executing it calls \verb|nextInstruction| in Listing \ref{lst:next_instr}, returning the instruction at the current program counter index, or \texttt{null} if there are no more instructions, or emulation has been stopped.

\begin{lstlisting}[caption=Next instruction method in the \texttt{Program} class, label=lst:next_instr]
public ExecutableInstruction nextInstruction() {
    if (this.stopped) return null;

    int instructionLoc = Math.floorDiv(pc.getValueAsInteger(), 4);

    if (instructionLoc > this.instructionList.size()-1) return null;

    return instructionList.get(instructionLoc);
}
\end{lstlisting}

The program counter is represented by the \texttt{ProgramCounter} class, which extends the \texttt{Register} class, adding additional functionality to jump and increment its value in order to allow conditional branching and sequential instruction execution.

Originally, execution called for the next instruction in the list by obtaining the index of the current instruction and adding one. This would of required branching instructions to directly interact with the emulation loop to permit branching. Thus, it followed to implement it like a physical processor would, wherein the next induction is the one at the location stored by the program counter. This allowed for a far easier to implementation of branch instructions.

During each instructions execution, the \texttt{ExecutableInstruction} wrapper will call the individual \texttt{Instruction} instances \verb|execute| method passing in the respective \texttt{InstructionOperands} instance. This results in the emulation of the instruction, which once complete triggers a call for the next instruction to be executed.

Emulation may be terminated at any point by calling \verb|stop()| which will cause \verb|nextInstruction()| to return null. This causes the recursive callback loop to return all the way back to the original \texttt{Program} \verb|execute()| call inside the \texttt{Emulator} class before finally terminating.

\section{Visualisation}\label{sec:impl_vis}
The visualisation of the project was the next major piece to implement. As stated in our design, JavaFX \cite{sunmicrosystems_2022_javafx} is used as our \ac{UI} library of choice, allowing for flexible building of user interfaces, as well as providing cross platform support out of the box. The production of the entire visualisation can be reduced down to two main sections User Interface (Section \ref{sec:ui}) and, Drawing and Animating Paths (Section \ref{sec:paths}). Additional \ac{UI} changes and adjustments made during the Integration in Section \ref{sec:impl_integ}.

\subsection{User Interface}\label{sec:ui}
Implementing the \ac{UI} was originally fairly complex. This was not due to our design, but as a result of poorly utilising JavaFX's \cite{sunmicrosystems_2022_javafx} tools. 

JavaFX \cite{sunmicrosystems_2022_javafx} provides two methods of creating interfaces. One way is to directly hard code these interfaces such as in Listing \ref{lst:hardcode_ui} to produce a basic layout. Originally this is how the \ac{UI} was being built. Whilst this worked, it was time consuming and required a lot of code to produce basic layouts and elements, whilst requiring extra code to ensure layouts and elements resized properly.

\begin{lstlisting}[caption=Example of hard coding grid layout with the resulting view in Figure \ref{fig:gridss}, label=lst:hardcode_ui]
GridPane gp = new GridPane();
ColumnConstraints side = new ColumnConstraints();
side.setPercentWidth(20);
ColumnConstraints side2 = new ColumnConstraints();
side2.setPercentWidth(20);
ColumnConstraints main = new ColumnConstraints();
main.setPercentWidth(60);;

RowConstraints row1 = new RowConstraints();
row1.setPercentHeight(70);
RowConstraints row2 = new RowConstraints();
row2.setPercentHeight(30);

gp.getColumnConstraints().addAll(side, main, side2);
gp.getRowConstraints().addAll(row1,row2);
\end{lstlisting}

\begin{figure}[h]
    \centering
    \includegraphics[width=\textwidth]{dissertation/DATA/grid layout.jpg}
    \caption{Grid layout produced from Listing \ref{lst:hardcode_ui}}
    \label{fig:gridss}
\end{figure}

Alternatively, JavaFX \cite{sunmicrosystems_2022_javafx} provides a templating language called FXML and a tool called SceneBuilder \cite{gluon_2022_scene} as discussed in the research section, allowing us to quickly build user interfaces via drag and drop.

Listing \ref{lst:fxml_example} shows part of our FXML code representing the Speed Slider and Clear, Reset and Run Buttons. Directly writing this FXML code is possible but would be very time consuming, and the equivalent Java code would be made up of creating new objects for every capitalised tag (VBox, Pane, ButtonBar, Button, Slider). Instead, by using SceneBuilder \cite{gluon_2022_scene} this section (Figure \ref{fig:fxml_view}) was built within minutes, with the underlying FXML being produce automatically.

\begin{lstlisting}[language=XML,morekeywords={VBox, Pane, ButtonBar, Button, Slider, AnchorPane, Insets}, caption=FXML code to generate the layout in Figure \ref{fig:fxml_view}, label=lst:fxml_example]
<VBox prefHeight="200.0" prefWidth="100.0" AnchorPane.bottomAnchor="0.0" AnchorPane.leftAnchor="0.0" AnchorPane.rightAnchor="0.0" AnchorPane.topAnchor="0.0">
   <children>
      <Pane fx:id="editorPane" prefHeight="778.0" prefWidth="381.0" />
      <ButtonBar prefHeight="40.0" prefWidth="200.0">
        <buttons>
          <Button fx:id="runButton" mnemonicParsing="false" text="Run" ButtonBar.buttonData="RIGHT" />
            <Button fx:id="clearButton" layoutX="412.0" layoutY="18.0" mnemonicParsing="false" text="Clear" ButtonBar.buttonData="LEFT" />
            <Button fx:id="resetButton" layoutX="412.0" layoutY="18.0" mnemonicParsing="false" text="Reset" ButtonBar.buttonData="RIGHT" />

        </buttons>
         <VBox.margin>
            <Insets left="5.0" right="5.0" />
         </VBox.margin>
      </ButtonBar>
      <Slider fx:id="animationSpeedSlider" blockIncrement="1.0" majorTickUnit="2.0" max="10.0" minorTickCount="1" showTickLabels="true" showTickMarks="true" value="1.0" />
   </children>
</VBox>
\end{lstlisting}

\begin{figure}[h]
    \centering
    \includegraphics{dissertation/DATA/implemented_slider_run_reset_clear.jpg}
    \caption{Visual render of implemented FXML in Listing \ref{lst:fxml_example}}
    \label{fig:fxml_view}
\end{figure}

As in Figure \ref{fig:fxml_view} the resulting visual layout is quite simple. However, it requires a large quantity of FXML to produce, mainly being all the positional wrapping elements such as AnchorPanes and VBox's. These Anchor panes allow us to anchor edges of elements to the parent causing them to fill the available space. Alternatively, VBox's controlling the flow of elements, with a VBox denoting vertical flow, and a HBox providing a horizontal flow..

\subsubsection{Controllers}\label{sec:impl_emu_controllers}
In order to interact with the FXML elements, JavaFX \cite{sunmicrosystems_2022_javafx} makes us of \texttt{Controller}'s. A controller is a class that contains the logic for interactive \ac{UI} elements.

To connect the FXML elements to our code, JavaFX \cite{sunmicrosystems_2022_javafx} provides the \texttt{@FXML} annotation, which is specified on a variable representing the same element type as the FXML element. For example Listing \ref{lst:fxml_anno} shows how we would obtain a reference to the Run button in Figure \ref{fig:fxml_view}, and then attach a handler it to run code to start the emulation.

\begin{lstlisting}[caption=FXML annotation for a button, label=lst:fxml_anno]
@FXML
private Button runButton;

/*...*/

runButton.setOnAction(e -> {
    resetHandler.run();
    runButton.setDisable(true);
    try {
        Emulator.getInstance().emulate(ca.getText());
    } catch (Exception ex) {
        this.showErrorBoxForException(ex);
    }
});
\end{lstlisting}

In order to allow for this connection the FXML file must be loaded, the controller attached, and the resulting package provided to the stage. The stage is the main \ac{UI} element shown to the user, with all elements being children of the stage. Listing \ref{lst:controller_loader} demonstrates how the application boots the \ac{UI} from setting the window name and position, and loading the controller (\texttt{MainWindowController}) and the FXML file (\texttt{main.fxml}).

\begin{lstlisting}[caption=\ac{UI} starting point, label=lst:controller_loader]
@Override
public void start(Stage stage) throws Exception {
    stage.getIcons().addAll(new Image(VisualStartingPoint.class.getResourceAsStream("/cpu-64x64.png")), new Image(VisualStartingPoint.class.getResourceAsStream("/cpu-32x32.png")));
    VisualStartingPoint.stage = stage;
    stage.setTitle("Visual RISC-V Simulator");

    FXMLLoader loader = new FXMLLoader(getClass().getResource("/main.fxml"));
    MainWindowController mwc = MainWindowController.getInstance();
    loader.setController(mwc);
    Parent croot = loader.load();
    mwc.init(stage);
    Scene sscene = new Scene(croot);
    stage.setScene(sscene);

    stage.setX(0);
    stage.setY(0);
    stage.show();

    Rectangle2D screen = Screen.getPrimary().getBounds();

    stage.setMaxHeight(screen.getHeight());
    stage.setMaxWidth(screen.getWidth());

    ModuleManager.getInstance().start();

    VisualStartingPoint.hostServices = this.getHostServices();
}
\end{lstlisting}

With this the \ac{UI} can be loaded and show to the end user, with all the interactive components linked up to the code for use by the emulator. With the final implemented design in Figure \ref{fig:final_implemented_design} taking into account our design considerations for colourblind users and visual impairments, as well as Nielson's Principles \cite{nielsen_2020_10}.

\begin{figure}[H]
    \centering
    \includegraphics[width=\textwidth]{dissertation/DATA/final_design.jpg}
    \caption{Fully Implemented \ac{UI}}
    \label{fig:full_impl_ui}
\end{figure}

\subsection{Drawing and Animating Paths}\label{sec:paths}
With the \ac{UI} layout implemented, the focus shifted onto producing the animation aspect of the project to allow for the full simulation of instructions.

\subsubsection{Drawing}
JavaFX \cite{sunmicrosystems_2022_javafx} doesn't directly provide a way to produce the kind of paths required for our animation. It does provide a \texttt{Path} class that produces paths. However, it includes functionality that wouldn't be used or required. Thus a custom Path class was produced that tracks a set of points, drawing lines between each set of points using the \texttt{Line} class as seen in Listing \ref{lst:lines}:
\begin{lstlisting}[caption={Example usage of the Line class}, label=lst:lines]
Line line = new Line();
line.setStartX(startX);
line.setStartY(startY);
line.setEndX(endX);
line.setEndY(endY);
\end{lstlisting}

With each point stored in a \texttt{Point} class holding the x and y value of each point. Each path itself is also given a name, this allows us to reference paths later when building up animation sequences, and also helps to identify them during debugging.

In order to simplify the creation of paths, a builder pattern was adopted in the form of a \texttt{PathBuilder} class (Listing \ref{lst:path_builder}), which starts with the Path's name and starting point, and then adding new points to the path, finally building the path once completed returning the newly created \texttt{Path} instance.

\begin{lstlisting}[caption=Path Builder, label=lst:path_builder]
public class PathBuilder {

    private final Path path;

    public PathBuilder(String pathName, double startX, double startY, HBox parent) {
        this.path = new Path(pathName, parent);

        this.path.addPoint(new PathPoint(0,0)); // We have to build a fake first path to get it to position correctly
        this.path.addPoint(new PathPoint(startX, startY));
    }

    public PathBuilder addPoint(double x, double y) {
        this.path.addPoint(new PathPoint(x, y));
        return this;
    }
    
    public Path build() {
        PathRegister.getInstance().register(this.path);
        return this.path;
    }
}
\end{lstlisting}

Due to the nature of building these paths and having them position correctly on screen, an additional invisible line has to be drawn for every path as seen in Listing \ref{lst:path_builder} on line 8. This was due to the fact that JavaFX \cite{sunmicrosystems_2022_javafx} defines (0,0) based on the first added line. Thus if a invisible line wasn't drawn, a line starting at 100,100 would appear in the very top left of the parent, instead of being at (100,100) as expected. Hence, drawing an invisible line first from 0,0 to our starting point alleviates this issue without impacting the visualisation.

With the builder created we can use it to produce all the paths as seen in Figure \ref{fig:final_implemented_design}, with the following code used to produce the line connecting the \ac{IR} and \ac{ID} as seen in Listing \ref{lst:path_builder}:
\begin{lstlisting}[caption=Example of use the path builder, label=lst:path_builder]
(new PercentagePathBuilder("IR_TO_ID", 20, 65, anim)).addPoint(36.25, 65).build().autoDraw(animationGroup, blue);
\end{lstlisting}

\subsubsection{Animating}
With our ability to draw paths, we now need to be able to animate on top of the to show the flow of data. JavaFX \cite{sunmicrosystems_2022_javafx} provides a few transition methods, with the main two being \texttt{TranslateTransition} and \texttt{SequentialTransition}, which allows us to translate elements, and sequentially run transitions respectively.

The logic for the animating over a specific path is contained within the \texttt{Path} class. Much like for drawing paths, we loop over all of the paths points, this time creating a new \texttt{TranslateTransition} for each pair of points. The transition is told to translate the given data element by the difference between the coordinates, and to animate this in a linear fashion as seen is Listing \ref{lst:translate}.

\begin{lstlisting}[caption=Translate Transition for two points, label=lst:translate]
TranslateTransition tt = new TranslateTransition();

/*...*/

tx =  (next.getX() * parentWidth - point.getX() * parentWidth);
ty =  (next.getY() * parentHeight - point.getY() * parentHeight);

tt.setByX(tx);
tt.setByY(ty);
tt.setInterpolator(Interpolator.LINEAR);
\end{lstlisting}

Due to the transition between each set of point varying in distance, the duration needed to be modified to ensure that the moving data element crossed each path at the same speed, instead of crossing each at different speeds. This was fixed by specifying a fixed speed for all animations of 100 pixels/second. Then by using the simple formula of $time = distance/speed$ the respective transition time could be calculated to ensure a constant speed over all paths.

Then each \texttt{TranslateTransition} could be added to the main \texttt{SequentialTransition}. Then when played, each \texttt{TranslateTransition} is played one after the other until all of them have played, or the animation is cancelled by the user.

In order to allow for speed control as specified in requirement \ref{req:speed}, we can adjust the rate at which the \texttt{SequentialTransition} plays by binding the value of the slider to the rate property as seen in Listing \ref{lst:slider}:
\begin{lstlisting}[caption=Slider linking code, label=lst:slider]
SequentialTransition sq = new SequentialTransition(circle);
sq.rateProperty().bind(MainWindowController.getInstance() .getAnimationSpeedSlider().valueProperty());
\end{lstlisting}
With a value between 0 - 10 allowing for pausing at 0, to a very fast animation with a value of 10.

With the ability to now animate over individual paths, a way to animate over multiple paths, as well as provide other actions need to be produces.

A custom class called \texttt{Animator} (Listing \ref{lst:anim_stage}) handles this for us. Again it follows a builder pattern to build up animations, with each section being denoted as a \texttt{AnimatorStage}, which is an abstract class, providing a \verb|play()| method to be implemented for the respective stage, as well as providing a convenient \verb|next()| method to call the next stage from within each stage.

\begin{lstlisting}[caption=Abstract \texttt{AnimatorStage} class, label=lst:anim_stage]
public abstract class AnimatorStage {
    protected Animator animator;

    public AnimatorStage(Animator animator) {

        this.animator = animator;
    }
    public abstract void play(StackPane circle);

    public void setAnimator(Animator animator) {
        this.animator = animator;
    }

    /**
     * Will run once he stage has finished, cleaning up itself and then calling animator.next()
     */
    public void after() {
        animator.next();
    };

}
\end{lstlisting}

The main stage is the \texttt{PathStage} which is provided with a path instance, and then runs the provided paths animation when the stage is run. The following stages also exists:

\begin{itemize}
    \item \texttt{RunStage} - Permits the running of any arbitrary code when the stage is executed,
    \item \texttt{InterpolateToPathStage} - Translates the visual data element between its current position and the start of the next path smoothly,
    \item \texttt{ALUStage} - Updates the visual elements of the\ac{ALU},
    \item \texttt{MemoryStage} - Updates the memory table,
    \item \texttt{TextChangeStage} - Updates the text on the visual data element.
\end{itemize}

Each of these stages can then be individually added via the \texttt{Animator} builder. The \texttt{Animator} then provides \verb|start()| and \verb|stop()| functions, as well as automatically registering all to a static list, to allow for the stopping of all animations at the same time, as the system supports parallel running of multiple animations at once.

For example we can use the code in Listing \ref{lst:animator_follow} to animate between the \ac{IR} and \ac{ID}:
\begin{lstlisting}[caption=Example of animating with the Animator, label=lst:animator_follow]
Animator().setText("I").followPath("IR_TO_ID").setText("D") .setText("D").followPath("ID_TO_CU");
\end{lstlisting}

\subsubsection{Fixed Elements}
Within the animation area in the centre of the screen, other than drawing path there are also fixed boxes to represent various components. 

These are then drawn to screen using the \texttt{VBox} element with a coloured background and then text on top. Each element exists as an extension of a abstract \texttt{VisualItem} class. This class does the heavy work of positioning the element on screen, as well as creating the default box structure with the text centred inside, as these are defined purely in code  and not with FXML. The class provides an abstract \verb|decorate()| method that allows each element to add additional styling or completely override the base styling.

For example the \ac{CU} overrides the base implementation to allow it to display more text to keep the user updated on what the system is doing, with a base method called \verb|updateCUContent| to specify the \ac{CU} content, with 5 methods sitting on top specifying the contents for the Fetch, Decode, Execute, Memory and Write cycle. (See Listing \ref{lst:cu_comp}).

\begin{lstlisting}[caption={\ac{CU} content updater methods}, label={lst:cu_comp}]
public void showFetch(ExecutableInstruction executableInstruction, int instructionMemLoc) {
    this.updateCUContent("fetch", executableInstruction, "READ", "LOC: " + instructionMemLoc);
}

/* ... showDecode, showExecute, showMemeory and showWrite ... */

private void updateCUContent(String stageName, ExecutableInstruction ei, String... options) {
    this.stageText.setText("STAGE: " + stageName.toUpperCase());
    this.options.getChildren().clear();
    this.options.getChildren().add(ComponentFactory.classText(ei.getInstruction() .getIdentifier() + " " + String.join(", ", ei.getOperands().getOperands()), "cu-option"));
    for (String option : options) {
        this.options.getChildren().add(ComponentFactory.classText(option, "cu-option"));
    }
}
\end{lstlisting}

\subsection{Callbacks and CompletableFutures}\label{sec:call_vs_cf}
Within the \texttt{Animator} we make heavy use of callbacks to control the flow of animations, with the end of one stage, calling for the next to be played, and so on. This is permitted via the \texttt{AnimatorFinishedCallback} interface, which simply allows us to encapsulate logic to be execute later. Alongside the callback functions of the \texttt{SequentialTransition} the whole animation sequence is played out.

Whilst this system works, it is prone to a issue of having to nest callbacks in places. For example in the code to run the 5 animation sections discussed in the Integration Section \ref{sec:impl_integ}, the following code is used to control the conditional execution of instructions with no \verb|execute()| handler as seen in Listing \ref{lst:cond_call}
\begin{lstlisting}[caption=Conditional callback execution, label=lst:cond_call]
decode.onFinish(() -> {
    if (execute != null) {
        ControlUnit.getInstance().showExecute(this, ProgramCounter.getProgramCounter().getValueAsInteger());
        execute.onFinish(() -> {
            ControlUnit.getInstance().showMemory(this);
            memory.start();
        });
        execute.start();
    } else {
        ControlUnit.getInstance().showMemory(this);
        memory.start();
    }
});
\end{lstlisting}
It contains a \verb|onFinish| call within another, which works, but isn't the nicest to write, and can become far more complex if more intricate animation sequences are needed.

A solution to this would be to make use of \texttt{CompletableFuture}'s. These allow us to wrap the callback operation into a more functional approach in which our code will wait for the async operation to complete before executing the next line, instead of specifying a callback to be executed later. 

The above callback code could be written as seen in Listing \ref{lst:pseduo} (pseudo code).
\begin{lstlisting}[caption=Functional pseduo code, label=lst:pseduo]
decode.start();
if (execute != null) execute.start()
memory.start()
\end{lstlisting}
This allows for the code to block until the current animation has finished and then moving onto the next line. This means no callback functions or async execution. This massively reduces the complexity of writing code to control the animation flow. On the other hand, it does require more additional design and setup to implement the \texttt{CompletableFuture} classes.

Listing \ref{lst:complt_future} shows an example of a \texttt{CompletableFuture} that just waits for 1 second before returning.

\begin{lstlisting}[caption=Example completable future code from CalliCoder.com \cite{singh_2022_java}, label=lst:complt_future]
CompletableFuture<Void> future = CompletableFuture.runAsync(new Runnable() {
    @Override
    public void run() {
        // Simulate a long-running Job
        try {
            TimeUnit.SECONDS.sleep(1);
        } catch (InterruptedException e) {
            throw new IllegalStateException(e);
        }
        System.out.println("I'll run in a separate thread than the main thread.");
    }
});

// Block and wait for the future to complete
future.get()
\end{lstlisting}

Currently the project doesn't make use of \texttt{CompletableFuture}'s, and this is something that will be refactored in in the future as it will require major breaking changes to the project code to implement.

\subsection{Percentages over Coordinates}
Within the implementation of paths an animations, path points were being stored as physical coordinates. Whilst this worked, it proved difficult to manually position the paths and box elements on screen. Coordinates had to be produces as a result of trial and error to position them perfectly. With this becoming much harder for ensuring paths overlaid on top of each other when they joined up, without producing a double thick path due to incorrect coordinates.

The intuitive solution was to switch to percentages. With percentages, each point was defined as a percentage of the width and height of the parent element, with (0\%,0\%) being in the top-left and (100\%,100\%) being in the bottom-right of the parent. This immediately made positioning paths and elements much quicker and easier, avoiding having to manually calculate element positions, leaving the application to automatically calculate these positions during run time.

Further this also allowed the \ac{UI} to be resizable instead of a fixed size. Originally, because of the fixed coordinate points, the animation area would remain a fixed size whilst the result of the \ac{UI} scaled as the user resized the application window, an original fix was to scale the parents contents, however this produced odd results with lines and elements colliding and not resizing correctly. However, when using percentages, the new positions of elements and lines can be recalculated based on the parent by simply multiplying the width/height by the percentage.

In order to implement this, the pathing classes had to be completely rewritten to support percentages. This mainly consisted of switching out the logic for calculating positions of points, from a fixed value, to a dynamically calculated value based on the parent as seen in Listing \ref{lst:perc_points}.
\begin{lstlisting}[caption=Percentage point caculations, label=lst:perc_points]
point.getX() * parentWidth

// Distance between current and next point:
tx = (next.getX() * parentWidth - point.getX() * parentWidth);
ty = (next.getY() * parentHeight - point.getY() * parentHeight);
\end{lstlisting}

To permit dynamic resizing, a resize listener was created to check for window resizes, and then call for the entire animation area to be redrawn instantly. This also called for any active animations to be hidden, and a new animation started at the new resized values, with the old animations being removed after. Due to the design of the animation system we can't simply reuse the active animation, as it pre-computes the values for each stage of the animation based on the current window size, and thus it would result in the animated element clipping through the resized elements, leaving paths of going off screen.

\begin{lstlisting}[caption=Auto draw function, label=lst:auto_draw]
public void autoDraw(Group drawGroup, Color clr) {
    this.parent.widthProperty().addListener((obs, oldV, newV) -> {

        removeLines(drawGroup);
        draw(drawGroup, clr);
        updateInprogressTransitionOnWindowResize();
    });
    this.parent.heightProperty().addListener((obs, oldV, newV) -> {

        removeLines(drawGroup);
        draw(drawGroup, clr);
        updateInprogressTransitionOnWindowResize();
    });
}
\end{lstlisting}

This is all handled by the \verb|autodraw()| function in Listing \ref{lst:auto_draw} and the \\ \verb|updateInprogressTransitionOnWindowResize()| function in Listing \ref{lst:resize_anim}.

\begin{lstlisting}[caption=Function to update inprogress animations on a window resize, label=lst:resize_anim]
private void updateInprogressTransitionOnWindowResize() {

    List<ActiveAnimation> toRestart = new ArrayList<>();
    toRestart.addAll(activeAnimations);

    for (ActiveAnimation activeAnimation : activeAnimations) {
        activeAnimation.getSequentialTransition().pause();
        activeAnimation.getSequentialTransition().setOnFinished(null);
        //activeAnimation.getSequentialTransition().stop();

    }

    activeAnimations.clear();

    toRestart.forEach(activeAnimation -> {
        StackPane activeCircle = activeAnimation.getCircle();
        EventHandler eventHandler = activeAnimation.getEventHandler();

        this.animate(activeCircle, eventHandler, activeAnimation.getSequentialTransition().getCurrentTime());
        activeAnimation.getSequentialTransition().stop();
        activeCircle.toFront();
    });

    toRestart.clear();
}
\end{lstlisting}

The automatic replaying of resized animations in significantly more complex than just re-drawing the lines and elements. This is due to needing to track all the active animations and their progress. Thankfully the \texttt{SequentialTransition} class tracks its progress through all the transitions and allows us to start the transition at any given point. 

Combined with simply tracking any currently active animation by individually tracking them in each \texttt{Path} class, the auto resizing can automatically stop any active animations and then immediately start new animations based on the newly resized windows, copying over the resulting callback functions and starting at the exact position the previous animation stopped for a very smooth experience with no tearing or odd behaviour.

In order to track an active animation, when a new animation is played a \texttt{ActiveAnimation} instance is created (Listing \ref{lst:active_anim}) which acts as a wrapper containing a reference to its parent \texttt{SequentialTransition}, the active animation element (referred to as the \texttt{circle}) and its callback event handler. It is then appended to a list of active animations that can be iterated through on a window resize to cause the animation resizing as discussed in the paragraph above.

\begin{lstlisting}[caption=ActiveAnimation wrapper, label=lst:active_anim]
class ActiveAnimation {
    private final SequentialTransition sequentialTransition;
    private final StackPane circle;
    private final EventHandler eventHandler;
    
    public ActiveAnimation(SequentialTransition sequentialTransition, StackPane circle, EventHandler eventHandler) {
        this.sequentialTransition = sequentialTransition;
        this.circle = circle;
        this.eventHandler = eventHandler;
    }
    
    public SequentialTransition getSequentialTransition() {
        return sequentialTransition;
    }
    
    public StackPane getCircle() {
        return circle;
    }
    
    public EventHandler getEventHandler() {
        return eventHandler;
    }
}
\end{lstlisting}


\section{Integration}\label{sec:impl_integ}
With the Emulator (Section \ref{sec:impl_emul}) and Visualisation (Section \ref{sec:impl_vis}) created, they need to be combined to work together in a tight integration. In the previous Visualisation section \ref{sec:impl_vis} various code listing hinted to this integration, mainly via the use of linking the visual \ac{UI} elements to the emulator code via the \texttt{@FXML} annotation and controllers \ref{sec:impl_emu_controllers}.

Our controllers exist in a hierarchy of the main \texttt{MainWindowController} starting by loading all the FXML annotations, and passing them to numerous sub-controllers that each handle the functionality of sections of the \ac{UI}. As a result there are 4 sub-controllers:
\begin{itemize}
    \item \texttt{VisualPaneController} - Handles management of all the path builders, and animated area,
    \item \texttt{CodeEditorController} - Handles the physical code editor and its functionality, as well as the buttons and slider below,
    \item \texttt{RegisterController} - Handles managing the register table and refreshing it,
    \item \texttt{MemoryController} - Handles managing the memory table and refreshing it,
    \item \texttt{SettingsController} - Handles loading and showing the settings pane,
    \item \texttt{AboutController} - Handles loading and showing the settings pane,
\end{itemize}

In the next 5 sub-sections, the modifications made to the emulator and visualisation to produce the tight integration are discussed, along with fixing a specific issue with register updating in sub-section \ref{sec:early_update_prob}.

\subsection{The 5 Animation Stages}
In order to provide a suitable animation sequence that is triggered by the emulation, it was decided that each instructions animation would be split into the 5 respective parts of the Fetch, Decode, Execute, Memory and Write cycle. This deviates from normal with most processors having 3 stages. RISC-V has 5 stages as RISC-V instructions are very simple and can be more easily pipe-lined in parallel with 5 stages. This way, common animations such as the fetching of an instruction from the instruction memory to the \ac{CU} could be written once, and then overridden if needed. This was permitted by amending the abstract \texttt{Instruction} class to include Fetch, Decode, Memory and Write methods along side the existing Execute method, as seen in Listing \ref{lst:fdemw}.

\begin{lstlisting}[caption={Additional Fetch, Decode, Memeory and Write methods added to the abstract \texttt{Instruction } class}, label=lst:fdemw]
public Animator fetch() {
    return new Animator() .setText(ProgramCounter.getProgramCounter().getValueAsHex()) .followPath("PC_TO_INSTR").setText("I").setText("I") .followPath("INSTR_TO_IR");
}

public Animator decode() {
    return new Animator() .setText("I").followPath("IR_TO_ID").setText("D").setText("D") .followPath("ID_TO_CU");
}

public Animator memory() {
    return new Animator();
}

public Animator write() {
    return new Animator();
}
\end{lstlisting}

With these additional stages, any individually instruction can override any of the animations as required, or extend them to include extra details or disable them entirely. These can then be executed linearly as seen in Listing \ref{lst:complt_future} in Section \ref{sec:call_vs_cf} via callbacks.

For more sophisticated animations and similar instructions, boilerplate code was refactored out into super-classes representing these similar instructions. For example the abstract \texttt{ALUCommonInstruction} class encapsulates the logic for instructions that take two registers and perform an arithmetic operation on them, providing a \verb|calc(int rs1, int rs2)| method than can be overridden by individual sub-classes such as the instruction class for \texttt{ADD} which can be seen in Listing \ref{lst:add_subclass}.

\begin{lstlisting}[caption={ADD Insutrction, making useof the \texttt{ALUCommonInstruction} class}, label=lst:add_subclass]
public class ADD extends ALUCommonInstruction {
    public ADD() {
        super("ADD");
    }

    @Override
    public int calc(int rs1, int rs2) {
        return rs1 + rs2;
    }

    @Override
    public String operator() {
        return "+";
    }
}
\end{lstlisting}

Further, the animations can also be abstracted away for these common instruction types. By abstracting this code out it become much more maintainable and follows the DRY (Don't Repeat Yourself) principle, with the code existing once, allowing for changes to reflect globally, instead of having to make the same change numerous times in different files.

This resulted in the creation of the \texttt{ALUAnimationBuilder} class.  This class provides a quick way to specify the animation for an arithmetic operation on tow register values, supplying the operator, registers and output. Internally it manages the creation of the required \texttt{Animators} to allow for an animation of data flowing into the \ac{ALU} from the registers and then the operation being applied, before finally sending it back to the registers. The builder provides two functions \verb|buildExecute()| and \verb|buildWrite| which return the respective \texttt{Animators} for the Execute and Write animation sequence. 

On top of the \texttt{ALUAnimationBuilder}, there is also a \texttt{CommonAnimationsFactory} class which provides quick access to an instance of the \texttt{ALUAnimationBuilder}, as well as static methods to provide quick generation of common animations such as moving data between the registers and \ac{ALU} and sending signals to various components, allowing much faster building of animation sequences instead of having to manually write the same lines over and over again, as seen in Listing \ref{lst:dry}

\begin{lstlisting}[caption=Factory method building a more complex animation moving data from the registers to the \ac{ALU},label=lst:dry]
public static Animator regToALU(Animator animator, Binary value, ALUConnection conn, boolean interp) {
    animator.setText(value.getValueAsHex()).followPath("REG_OUT");

    if (interp) {
        animator.interpolateAndFollowPath("REG_OUT_TO_RM-JUNCTION");
    } else {
        animator.followPath("REG_OUT_TO_RM-JUNCTION");
    }

    if (conn == ALUConnection.IN_1) {
        animator.followPath("ALU_IN_1");
    } else {
        animator.followPath("ALU_IN_2");
    }

    return animator;
}
\end{lstlisting}

\subsection{Memory and Registers}
The registers and memory both require a suitable representation on screen, with a table being the most ideal. Currently both register and memory values are just simply dumped as text onto the screen. Thankfully JavaFX \cite{sunmicrosystems_2022_javafx} provides a element called \texttt{TableView} which allows us to generate tables and fill them with data that can be updated in real time.

For both the registers and memory the tables being as an FXML element with the 4 columns (Register/Location, Denary, Hex and Binary) already pre-set, but with no data. Within the \texttt{RegisterController} and \texttt{MemoryController} references are obtained to both the table element and each individual column. With this we can specify a value factory for each column, these value factory will automatically pull a value from the passed class. They do this by us specifying the name of the function that returns each respective value, omitting "get" from them, in this case we can see in Listing \ref{lst:table_view} that for the location, denary, hex and binary we specify the respective function name on the \texttt{Register} class. Now when passing a list of \texttt{Register}'s from the \texttt{RegisterSet} instance, the table will automatically fill each row with the required data.

\begin{lstlisting}[caption=\texttt{RegisterController} code to obtain column values, label=lst:table_view]
location.setCellValueFactory(new PropertyValueFactory<>("Name")); // will call Register#getName()
denary.setCellValueFactory(new PropertyValueFactory<>("ValueAsInteger")); // will call Register#getValueAsInteger()
hex.setCellValueFactory(new PropertyValueFactory<>("ValueAsHex")); // will call Register#getValueAsHex()
binary.setCellValueFactory(new PropertyValueFactory<>("Value")); // will call Register#getValue()
\end{lstlisting}

The memory operates the same for attaching column values, however both controllers make use of slightly different mechanisms to refresh the table data. The register table always maintains a list of the 32 base register, calling for the register values to be re-fetched on refresh. On the other hand due to memory's dynamic nature, each refresh makes a call to the \texttt{Memory} instance, to obtain a fresh list of any available memory cells, automatically sorting them via the location in an ascending nature.

\subsubsection{The Early Update Problem}\label{sec:early_update_prob}
The table views work exceptionally well with no issues themselves. However, during testing a problem in which when writing to a register, the value in the table would update before the write animation had finished. This caused some confusion as it implied that any register changes are written instantly, which may seem the case in the real world, but there will be a small quantity of time of which the data is travelling, which the simulator should also display.

The root of the problem was due to the writing of register values being separate to the animation sequence. Register values are set via a \verb|write| class on the \texttt{RegisterSet} class. However, this executes instantly which originally wasn't considered an issue with registers updating once each instruction has finished executing. Unfortunately, due to the callback nature, this refreshing was often called mid way through execution resulting in the emulation being one induction ahead of the visualisation, resulting in register updating before the animation had ended.

Two fixes were identified, the first including moving register writing into a animation stage, however this proved to be tricky to implement and didn't work consistently enough to be viable. The second was to buffer register writes, and make use of an animation stage to then flush them through the system calling a refresh after. This worked perfectly, with only minor changes required to the \texttt{RegisterSet} class as seen in Listing \ref{lst:reg_buffer}. These included: 
\begin{itemize}
    \item Adding a set to act as a buffer of changes (this won't cause missed writes, as it is flushed per instruction),
    \item Modifying the write to push writes to the buffer instead of immediately writing to the registers,
    \item Adding a \textbf{refresh()} function, that iterates over the write buffer, writing the changes to the respective registers.
\end{itemize}

\begin{lstlisting}[caption={Refresh mechanism, writing buffered register values}, label=lst:reg_buffer]
private final Set<Register> toUpdate;

public void write(String registerName, String value) throws RegisterNotFoundException, InvalidValueException {
    if (ignoreZeroRegister(registerName)) return;

    // Check the given value is infact binary
    Utils.isBinary(value); // throws its own error

    Register cloned = this.load(registerName).clone();

    cloned.setValue(value);

    this.toUpdate.add(cloned);
}
    
public void refresh() {
    toUpdate.forEach(reg -> {
        this.registerMap.get(reg.getName()).setValue(reg.getBinary());
    });
    toUpdate.clear();
}
\end{lstlisting}

\subsection{Pushing updates}
Within some of the visual animation components, it was imperative that their text updated during animations. This required a way to push updated information to them. A common way to create this functionality is creating an even/bind. For the project, a generic \texttt{Bind} interface was created which simply provides a \verb|execute(E value)| method. Generics have been made us of here so that a bind may update any kind of value. 

A bind is created by simply creating a new instance of a bind and implementing the \verb|execute(...)| method. It can then be called by running \verb|bind.execute(...)|. This allowed for the simple updating of the \ac{ALU} text with Listing \ref{lst:bind} showing how the a bind is created to provide a \texttt{Binary} instance and set the respective text value as well as how the \texttt{ALUStage} calls the bind with a \texttt{Binary} instance.

\begin{lstlisting}[caption=\ac{ALU} bind creation and execution, label=lst:bind]
this.inBind1 = (binary) -> {in1.defaultText.setText(binary.getValueAsHex());};
this.inBind2 = (binary) -> {in2.defaultText.setText(binary.getValueAsHex());};
this.outBind = (binary) -> {out.defaultText.setText(binary.getValueAsHex());};

...

ALU alu = ALU.getInstance();
alu.getInBind1().execute(this.in1); 
alu.getInBind2().execute(this.in2);
\end{lstlisting}

Thanks, to this simple interface it makes pushing data relatively simple, and due to the generic nature we could of also specified the bind take a string value, an integer or even a \texttt{Register} instance itself.

\subsection{Code Editor}
After creating the visualisation, the method of inputting RISC-V assembly was rather basic, with just a simple text box. It would be more intuitive and friendly for the code editor to have similar features to other simulators and some integrated development environments. Including basic syntax highlighting and line numbers to make debugging easier.

JavaFX \cite{sunmicrosystems_2022_javafx} has a convenient library called RichTextFX \cite{fxmisc_2023_fxmiscrichtextfx}, which provides a \texttt{CodeArea} class which allows us to produce a customised code editor with support for syntax highlighting and line numbers.

To allow for syntax highlighting RichTextFX \cite{fxmisc_2023_fxmiscrichtextfx} requires us to compute and apply styled spans which apply a set of styles to a specific area of text based on the index of the first character and the length.

These spans are computed using regex. First a pattern is compiled that recognises all the loaded instruction names and any comments (starting with a \%). It is then matched against the entered code, with the start and length of each match being stored alongside the respective style class to be applied. These are then returned to the \texttt{CodeArea} instance, that applies the spans resulting in the styling appearing.

Styles in this case mimic web-based Cascading Style Sheets, with a class encapsulating styling properties, that are applied to each matched element. In this case we have the \verb|.instr| and \verb|.comment| classes which make instruction names purple and bold, and comments green alike as seen in Figure \ref{fig:syntax_high}.

\begin{figure}
    \centering
    \includegraphics[width=0.6\textwidth]{dissertation/DATA/syntax_high.jpg}
    \caption{Example of applied syntax highlighting}
    \label{fig:syntax_high}
\end{figure}


\section{Module System}\label{sec:impl_mod}
Later in development, when the stage of implementing additional RISC-V \cite{fxmisc_2023_fxmiscrichtextfx} extensions was reached, a modular approach was decided upon as stated in the Design in Section \ref{sec:module}.

To re-iterate from the start of this section: We refer to modules as the encapsulation of logic that allows us to implement a RISC-V extension such as Multiply and Divide.

Modules exist as a implementation of the \texttt{VRVSModule} interface in Listing \ref{lst:vrvs_interface}. Every created module must have their main class implement the 3 provided methods, to allow for the main application to identify each module, and also enable/disable it.

\begin{lstlisting}[caption=\texttt{VRVSModule} interface for creating new modules]
public interface VRVSModule {
    String getName();

    void onEnable(VRVSApi api);

    void onDisable(VRVSApi api);
}
\end{lstlisting}

In the \verb|onEnable(...)| and \verb|onDisable(...)| methods a instance of \texttt{VRVSApi} is provided. This instance provides methods to allow modules to extend the base system, providing a method to add new instructions, add additional code examples and even add additional register panes so that more registers other than the base 32 registers can be seen.

Each \texttt{VRVSApi} instance is specific to a module. This was done to enable the tracking of which module added features via the API, meaning those additions can easily be unloaded by removing all features tagged against the module being disabled. This is done by allocating multiple sets which store references to the added items. These sets can then be iterated over in order to remove all the added items. A set is used here to remove the possibility to adding duplicate items, as a set doesn't permit duplicates, thus if a user were to register the same instruction twice, it wouldn't then attempt to remove it twice, as the internal instruction manager will reject duplicate instructions anyway.

\subsection{Loading Modules}
In order to allow for modules to be loaded, the Java \texttt{ServiceLoader} is used. The \texttt{ServiceLoader} allows us to dynamically load external classes in to the Java Virtual Machine at run time. The loader first requires a set of URL's to load from, in this case we can provide the URL to the module's compile jar file which for example might be called \texttt{Module.jar}. We pass this URL to the \texttt{ServiceLoader}, specifying the class to be loaded. In this case the target class is \texttt{VRVSModule}. As a result, any class implementing \texttt{VRVSModule} will be loaded from the Jar, or list of Jar's given.

These loaded classes are then iterated over and wrapped in a \texttt{LoadedModule} instance, and then finally are checked to see if the user has specified if the respective module is enabled or not, defaulting to being enabled.

However, because of using the \texttt{ServiceLoader} each module is considered a service, and must contain a service identifier that is used to pull the required \texttt{VRVSModule} class. Within the resources section of a modules project, a directory structure of \verb|META-INF.services| must be created, with a file inside with the name of the full \texttt{VRVSModule} class path within the main application, and then each line inside containing the full class path of the modules class implementing the \texttt{VRVSModule} class.

The \texttt{LoadedModule} wrapper exist to provide extra functionality to handling modules that can't be directly inside the module class itself, otherwise it would open it up to modification by module creators. The wrapper provides methods to call the modules enable and disable methods, passing in the modules specific instance of the \texttt{VRVSApi} which is stored inside the \texttt{LoadedModule} wrapper. The wrapper also provides a method to check if a module is enabled, and also allow us to specify if a module is built-in and packaged with the core application, and if so, that this module cannot be deleted, only disabled.

All of these \texttt{LoadedModule}'s are managed by a \texttt{ModuleManager} instance, which holds a reference to every module, and provides functions to load internal and external modules, delete modules and ensure that the required directory structure exists for storing modules. It further provides a simple way to initiate the module system by simply instantiating the \texttt{ModuleManager} class and calling \verb|start()|, and then completely cleaning up on application close by calling \verb|end()|.

\subsection{\ac{UI} Additions}
Currently modules can only be added or removed by manually placing them into the created modules folder on a users system. Thus a easier way to manage modules via the application was implemented through the use of a popup window.

As based on its design the window contains a table listing the loaded modules, and if they are enabled or not, with buttons below to add and delete modules, which can be seen in Figure \ref{fig:module_popup}.

\begin{figure}[H]
    \centering
    \includegraphics[width=0.5\textwidth]{dissertation/DATA/module_popup.jpg}
    \caption{Implemented Module Popup}
    \label{fig:module_popup}
\end{figure}

The popup is designed using FXML residing in the \texttt{modules.fxml} file, with interactivity added via the \texttt{NewModulePopup} controller. Each row of the table contains the modules name and a checkbox to enable and disable the module. On toggling a module a popup is show to confirm the actions completion.

Additionally, the very first column of the table has he option to display a checkbox for externally added modules. This permit the user o select one or more modules and then click delete. This will ask the user to confirm their decision before the module system unloads and deletes the specified modules.

\section{Additional Features}
\subsection{File Saving/Loading}
The majority of code emulators and simulators permit the saving and loading of user written code, and so shall the project.

This can be quickly implemented by adding menu-bar options to load and save a file, and then implementing the feature inside the \texttt{CodeEditorController} class.

For both loading and saving we make use of the Operating Systems file chooser, via JavaFX's \cite{sunmicrosystems_2022_javafx} wrapper. This wrapper is known as \texttt{FileChooser} and it allows us to specify the default file extension to use as well as open two individual dialogues to allow for retrieving a file and saving a file.

The open dialog returns a \texttt{File} instance, that can be read using builtin file utilise, with its contents being pasted directly into the code editor. For saving, the dialog returns a \texttt{File} instance representing the saved location on the system. Then using a \texttt{PrintWriter} the contents of the code editor an be written to the file and saved as seen in Listing \ref{lst:file_save}.

\begin{lstlisting}[caption=File saving implementation using \texttt{FileChooser}, label=lst:file_save]
FileChooser fileChooser = new FileChooser();
fileChooser.getExtensionFilters().add(new FileChooser.ExtensionFilter("RISC-V", "*.riscv"));

fileSave.setOnAction(e -> {
    fileChooser.setTitle("Save");
    File file = fileChooser.showSaveDialog(VisualStartingPoint.getStage());
    if (file == null) return;
    PrintWriter writer = null;
    try {
        writer = new PrintWriter(file);
        writer.print(ca.getText());
        writer.close();
    } catch (FileNotFoundException ex) {
        new Alert(Alert.AlertType.ERROR, "Failed to save to file!").showAndWait();
    }
});
\end{lstlisting}

\subsection{Animation Toggling}
In some cases a user may not want to simulate their code, and instead just emulate it jumping straight to the end results. To allow for this a toggle was implemented which disables the execution of animations.

This required slight modification to aspects of the emulation code, mainly by passing in the animations toggle into the \texttt{Program} instance, which in turn passes it to each \texttt{ExecutableInstruction} instance. Then when each \texttt{ExecutableInstruction} is executed, if animations are disabled it will only call the referenced instructions \verb|execute(...)| method, ignoring the fetch, decode, memory and write animations. Immediately calling for the execution of the next instruction, ignoring the returned Animator from the \verb|execute()| method as well.

Due to changes to fix the early register updating problem, a check also had to be added here to override this fix for disabled animations, otherwise register writes would never hit the actual registers, causing all operations to result in 0. This modification can be seen in Listing \ref{lst:noAnimModifi}.

\begin{lstlisting}[caption={Modification to register \texttt{write(...)} for instance writing when animations are disabled}, label=lst:noAnimModifi]
if (!SettingsController.areAnimationsEnabled()) {
    Register reg = this.load(registerName);
    reg.setValue(value);
    return;
}
\end{lstlisting}

Once the entire execution is completed, the registers are manually updated by the emulator to show the end results.

\chapter{Testing \& User Feedback}
\label{ch:testing}
\section{Testing}
Testing has been ongoing during the development of the project, being split into two sections: Test Driven Development and Manual Testing. Testing has been essential in order to ensure that the project code runs as expected without issue as code was written and refactored down the line. It permitted the ability to check that code executed as expected, and as a result has allowed for a multitude of bugs to be fixed. Ensuring that the end user receives a polished application with no unintended behaviour.

\subsection{Test Driven Development}
Test driven development came first encompassing the implementation of the emulator. Tests were written in Java \cite{sunmicrosystems_2022_java} using JUnit 5 \cite{junitteam_2019_junit}, which provides a suite of tools to implement and run a test harness. In our case this was implementing simple unit tests for the respective emulator components. A test is denoted via the \verb|@Test| annotation, calling once of JUnit's \verb|assert| functions. These test a provided result against the expected value, passing or failing respectively. With a simple example in Listing \ref{lst:junit_example} which asserts that $1+1=2$.

\begin{lstlisting}[caption=JUnit test example, label=lst:junit_example]
@Test
void exampleTest() {
    assertEquals(2, 1+1);
}
\end{lstlisting}

Before the implementation of each major component (Registers, Memory, Instructions and Parser within the emulator), a set of tests were written encapsulating the expected behaviour. Then each component was implemented, with tests being run as changes were made. Slowly tests began to pass and fail as the implementation was completed and bugs were fixed. This ended with all the tests passing.

This format of testing proved especially effective for Instructions. It enabled test writing following the RISC-V specification \cite{riscv_2015_riscv} to produce tests that appropriately check that each implemented RISC-V instruction emulates the physical instruction correctly. Each instruction could then be written and tested, with changes made for any failing tests.

\subsubsection{Automatic Testing}
As a result of using JUnit \cite{junitteam_2019_junit}, the option to automate testing on a version control push to GitHub \cite{github_2013_build} via Git \cite{git_2022_git} was available by making use of workflows.

Simply, the GitHub repository listens for pushes, and on each push launches a workflow in which a list of tasks are run on a virtual machine. For this project, it simply builds the jar and then runs the test handler, returning a visual document listing all the tests and whether they passed or failed. An example output of the workflow can be seen in Figure \ref{fig:github_test_output}, with failing tests including an error stack trace to help with debugging.

\begin{figure}[h]
    \centering
    \includegraphics[width=0.8\textwidth]{dissertation/DATA/githubtest.jpg}
    \caption{Github test runner output}
    \label{fig:github_test_output}
\end{figure}

\subsubsection{Issues}
Test driven development was not without its issues. In some cases a few false positives occurred in which tests passed or failed when they shouldn't off. Fortunately, this was due to mistakes in test writing, with some test logic not calculating the correct value for the test to test against. 

A few issues also arised with the GitHub test runner. Locally the tests executed in an order that loaded instructions and cleared register and memory values in certain tests, but not all. However, the GitHub test runner ran the tests in a different order causing values to exist in the memory and registers between tests. This resulted in cascading test failures, with each additional test causing the next to fail and so on.

Thankfully, this was a simple fix to ensure that between test values were cleared completely. However, another issue quickly arose relating to the test for reading a hexadecimal value into a register. This test passes locally with correct execution, but fails on the Github runner. Specifically, the value stored in the register is expected to be \texttt{b} (11), but instead a value of 7 is returned. Despite considerable debugging and modifications, this test continues to fail on the automated runner. Whilst it is important for tests to all pass, it is the case that this test will simply be ignored as we can manually verify the underlying logic operates as expected. It is likely that the test will be removed in the future or rewritten such that it passes on the runner on Github as well as locally.

\subsection{Manual Testing}
Not all parts of the application can benefit from automated testing. The visualisation requires the physical pressing of buttons and inputting code. There may be applications that provide this functionality for Java. However for our simple \ac{UI}, a manual approach was suitable.

This approach consisted of repeatedly testing the \ac{UI} as elements were implemented. Ensuring that code ran on button clicks, animations played out smoothly in the right order, as well as ensuring elements resized properly. This testing was slower, but proved to be highly useful in order to spot visual discrepancies with resizing the window, and for spotting smaller issues with the animation sequence. Such as the animated element passing under the lines, or pushing the entire animation area off screen due to a mistake in positioning.

To ensure manual testing was rigid, a procedure was followed. This consisted of interacting with the \ac{UI} in an order in which all buttons were pressed, then menus were pressed followed by testing text inputs. Finally finishing with a cohesive script to test that all animations played correctly at varying speeds ensuring no anomalies occurred.

Further, more destructive testing was performed, with the aim of intentionally trying to break the application. This testing was performed as a worst case scenario, in which a user may perform a combination of actions that break the application, either by accident or intentionally. This testing was sporadic with no set procedure. 

A bug that was fixed as a result of random testing occurred from the spamming of enabling and disabling of modules. If toggled fast enough it was possible for a module to be part loaded or unloaded during the next toggle which would result in the program producing an error and in some cases crashing. This was an error that would of otherwise have gone unnoticed until reported by a user, and was quickly fixed by locking out the toggling checkbox until a load/unload is finished.

\section{User Feedback}
Alongside testing, user feedback was also important. It ties into our testing providing a wider set of testing environments from different devices and operating systems, as well as different interaction styles based on individual users. 

To allow for user testing a version of the final project application was packaged into a distributable Jar file that could be run on any system with Java 19 \cite{sunmicrosystems_2022_java} installed. Further, to provide an easier installation experience for fellow department students, a quick installation script was produced that downloaded the Jar and then enabled Java 19, followed by launching the application.

\begin{figure}
    \centering
    \includegraphics[width=0.8\textwidth]{dissertation/DATA/gform.jpg}
    \caption{User feedback google form}
    \label{fig:gform}
\end{figure}

Feedback was then collected via a google form (Figure \ref{fig:gform}) simply asking for what users liked, disliked and what they would find beneficial to have added.

The resulting user feedback was positive, with praise for the ease of use and functionality. The feedback also provided ample suggestions for change and improvements, as well as identifying issues with the animation sequences and elements sizing incorrectly.

For example one user stated: "The U of ALU goes outside the orange shape and merges with the white background". This occurred when making the application widow smaller, as a result of poorly implementing the \ac{ALU} element. The element contained 3 child text elements that were spaced inside the image of the \ac{ALU}. However, as a result of doing this, when the text values holding \ac{ALU} input and output hexadecimal values became large. This caused the \ac{ALU} image to stretch in order to fit the text, which caused the image to become distorted either becoming too wide or too tall, with the white text clipping onto the background becoming hard to read.

In order to fix this issue, the text elements were separated out of the image element, being positioned individually on top of the image element, and not as children. This way the \ac{ALU} image could resize properly with the text moving respectively. Further as an additional safety measure, the text was made black, so that if it managed to overflow, it would be clearly readable.

This user feedback was invaluable to the project, without it many of these issues wouldn't of been caught and fixed, and it also provided a mental boost during the end of development where motivation was dipping.


\chapter{Project Management}
\label{ch:project_management}

Appropriate project management is key to any projects success, it provides the core work flow and stages for a project giving structure to work against to ensure goals are achieved, with progress tracked and checked. Below the Methodology, Timetable, Risk Management and Ethical, Legal, Social and Professional Issues for this project are discussed.

\section{Methodology}
The project followed an Agile \cite{atlassian_2022_what} approach, with a few plan driven aspects. Within this approach a backlog was generated using Notion \cite{notionlabs_2023_your} as a to-do list as seen in Figure \ref{fig:notion}. This to-do structure permitted the ability to create sub tasks, as indented lines, such that sub tasks could be completed independently of the parent task.

\begin{figure}
    \centering
    \includegraphics[width=0.8\textwidth]{dissertation/DATA/notion-todo.jpg}
    \caption{Backlog creates as a to-do list in Notion}
    \label{fig:notion}
\end{figure}

It may of been more appropriate to utilise the backlog within Github since it was being used for version control and could be tightly integrated. However, the simpler approach of using Notion was chosen as it was simpler to implement, modify and use, compared to creating lots of individual tasks on GitHub.

Only a handful of plan driven aspects existed, mainly being the overarching concepts of splitting the project into 3 core parts being the Emulator, Visualisation and eventually a module system, with a repetitive agile approach followed within each concept. In this case each individual element was continuously developed in increments with a consumable product available during the development cycle for testing and use.

A notable part of the agile approach was applying our ability to re-plan and re-design mid implementation. For example with the switch to percentages for animation, which required a new plan and design for the system. Another bigger example being the decision to implement the module system, which required mid project planning to identify how it would effectively work and integrate, followed by its individual design and implementation.

These major changes wouldn't of been as feasible if we had foe example followed a Waterfall methodology \cite{ganttchartsoftware_2023_waterfall}. This would of required restarting the core methodology loop, in order to make these changes as waterfall focuses on linear progress with a physical product available at the end of the development, whereas agile permits a quicker production of a viable product

\section{Timetable}
The project followed a weekly timetable split into 4 sections: Term 1, Term 2, Christmas and Easter as seen in Figures \ref{fig:tt_t1}, \ref{fig:tt_t2}, \ref{fig:tt_christmas} and \ref{fig:tt_easter} in Appendix \ref{app:timetable} respectively.

The timetables incorporate a weekly task, promoting a continuous linear flow of progress, with included breaks and catch-up weeks to ensure that delays were accounted for, and to ensure that burnout delays could be accommodated for. 

This timetable was followed mostly, with a few deviations including starting the next weeks content earlier, or delaying work into another week due to a higher course load. However, due to the efficient planning of the timetable, it proved useful to ensure productive use of time. It was also decided to plan the Christmas and Easter breaks to ensure work on the project continued, whilst allowing for more free time within these periods to relax and refresh.

A few alteration were made to the timetable towards the end. These included the removal of a scheduled holiday break, and the addition of the user feedback section. Further the allocation of time for the presentation was massively over-judged, allocating 5 weeks. Instead approximately 3 weeks were assigned to produce the slides, create the script, practice and finally present.

\section{Risk Management}
The project has been mainly risk free. The only main risk considered was the failure to produce a minimal viable product via the end of the Christmas break. However, this risk was quickly dismissed with a minimal viable produce ready post Christmas.

The only additional risk to consider was the possible loss of the code base due to unforeseeable circumstances. However, via the use of version control with \cite{github_2013_build} and Git \cite{git_2022_git}, as well as creating additional remote copies on the department system and zipped backup on google drive, the possibility of complete code loss has been made negligible, thus dismissing this risk.

\section{Ethical, Legal, Social and Professional Issues}
As a software development project very few Ethical, Legal, Social or Professional issues are present. No social or ethical issues exist, as no personal data has been collected or processed, and neither does the project cover any problematic content or areas of society.

There are a few considerations for legal and professional issues however, which are discussed below.

\subsection{Legal}
Java is not always free to use, dependent on the distribution used and its respective licence. In this respect Java is free to use for users, but may incur a cost for developers depending if they make use of additional paid options.

Thankfully, the project makes no use of paid features, and further makes use of the OpenJDK 19 \cite{oraclecorporation_2022_openjdk} distribution which has much fairer and lax rights usage compared to the OracleJDK and is commonly used for open-source projects.

Another consideration is the storage of user feedback data. Whilst no personal information was collected, it is still important to follow the General Data Protection Act \cite{theeuropeanparliamentandthecounciloftheeuropeanunion_2016_regulation} to ensure collected data is appropriately used, stored and then destroyed, with a notice to users visible. 

By using Google Forms, the collected data is only visible to the project, and can be easily destroyed, with the ability to easily display a GDPR notice on the form itself.

\subsection{Professional}
In a professional capacity, the project will need to ensure that standard principles are followed and maintained. Such as professional commenting and producing maintainable code. Further, ensuring that the resulting simulator is produced to a high standard, with no unnecessary gimmicks or alterations that may deter usage.

\chapter{Evaluation}
\label{ch:evaluation}
It is important to evaluate the project against our original requirements in Chapter \ref{ch:requirements} and expectations to determine the outcome of the project. We'll start by evaluating our completion of the original requirements, then the simulators 3 main components of Emulating, Visualising and the Module System, then evaluation the project management and finally discussing some limitations of the project.

\section{Requirement Evaluation}
The majority of requirements have been met and implemented, with a few lower priority requirements not met.

\subsection{Emulation Requirements}
For the emulation, requirements \ref{req:e1}, \ref{req:e2}, \ref{req:e3}, \ref{req:md}, \ref{req:fp} and \ref{req:e7} have been fully completed with the full functionality found within the emulation of RISC-V code.

Requirements \ref{req:e2} and \ref{req:e3} are fully implemented, however they could benefit from additional work to provide more comprehensive lexical and semantic analysis with additional feedback available for specific cases, as well as possibly including more highlighting for incorrect syntax and grammatical errors.

Unfortunately \ref{req:e4} was never implemented. It was specified as a "Should" requirement meaning it wasn't originally considered as a required feature. However, this is something that can be included later and postponed to the future work section (Section \ref{sec:future_work}).

\subsection{Visualisation Requirements}
Out of the 3 core visualisation requirements \ref{req:v1} and \ref{req:v2} have been completely implemented, with a fully functional \ac{UI} presented to the user and the visualisation of data and instructions flowing about the system complete.

Requirement \ref{req:v3} specified more details into the parts of the visualisation system. Of these 5 sub-requirements \ref{req:v3_a}, \ref{req:v3_b} and \ref{req:v3_d} have been fully implemented with data visibly moving around the system, addressing request being sent before reads and writes and animation speed being controllable via the implemented slider.

Sub-requirement \ref{req:v3_c} has been partially implemented, with the ability for values to be manipulated in parallel available, but not utilised within the simulation, instead choosing a more linear approach to keep things simple and uncluttered. On the other hand, Instruction stepping \ref{req:v3_e} was not implemented at all. 

Instruction stepping is a more advanced feature that is possible to add with the current implementation, however the focus of the project shifted onto building a robust simulator with a consistent linear simulation sequence. As a result of the slider implementation, it is is possible to artificially create a stepping feature by manually pausing the animation after each instruction as executed. With the requirement being denoted as a "Could" it was also much further down the list of priorities, and like requirement \ref{req:e4}, it can be included as a later addition for future work.

\section{Component Evaluation}
During its development, the project has always been split into three independent components being the Emulator, Visualisation and then the Module System. As the nature of each section differs, all three will be discussed individually below:

\subsection{Emulation}
The resulting emulation system is very robust and concrete. Due to the the well created design and following implementation, the emulator works as expected and produces a correct emulation of all the implemented RISC-V \cite{riscv_2015_riscv} instructions. Further it gracefully handles errors from a wide array of sources, from lexical errors, to internal emulation logic errors arising at any point.

The emulation is also well produced, with a simple way to interact with it outside of the visualisation. Thanks to this, external systems may easily implement and emulate RISC-V code without having to hook into the front end. Instead they can directly hook into the emulator. This was not originally intended, but exists as the result of careful planning and execution resulting in this additional feature.

\subsection{Visualisation}
The visualisation is complete, but is a part of the project that can be continuously iterated upon indefinitely to add new features and abilities as well as streamlining existing features.

The current visualisation is well presented and easy to follow, providing a simple and intuitive understanding of the processor flow, without providing too much information at once. From the start to the end of the project, the visualisation layout has remained fairly consistent with a few major changes and considerations for accessibility, with the resulting implementation perfect for the use case of the simulators success.

\subsection{Module System}
The Module system was somewhat an afterthought for the project, materialising part way through the project. Despite this it has played an imperative part in making the project extendable beyond initial expectations. The system provides a simple way for anyone to extend the simulator in both emulation and visualisation aspects without restrictions to just the project developer.

As a result of the module system the original RISC-V extensions were successfully implemented producing a more well rounded experience which is far better than what was originally intended. 

As a result the Module system has been an important addition to the project and should of been considered from the very beginning instead of part way through the project. 

\section{Project Management}
The execution of the project has been relatively flawless. The choice of following the Agile \cite{atlassian_2022_what} methodology has proven to be the correct choice. The backlog of tasks allowed more important aspect to be complete first, with less important tasks moved lower down, with the ability to re-prioritise with ease.

Through the following of the planned timetable, continuous development was performed with a clear layout as to what to work on each week. Combined with weekly/fortnightly supervisor meetings, the project flowed well with the ability to discuss designs, implementation and run through decisions. Whilst suggesting changes and providing critical feedback to help steer the project in the right direction.

Further, the use of testing allowed for faster development, ensuring that features worked as expected. With pre-written tests removing the issue of having to preform extensive debugging to identify errors later on, whilst providing constant feedback ensuring that newer features or revisions didn't cause breaking changes.

\section{Limitations}
The project is not without its limitations. Thankfully only a few limitations have been identified, all of which can be overcome in time:
\begin{itemize}
    \item The implementation of the base 32 bit instructions set is not complete, missing the Jump \& Link, Synch and Environment instructions, as well as all Control Status Register operations. This decision was made to provide a simpler set of instructions for newer users, and reduced the complexity of the overall emulation. However it is possible to extend either the base system or create an extension module to encapsulate these missing instructions in the future.

    \item The module system exposes the internal classes to external developers creating custom modules. Whilst not detrimental, this should be avoided by providing developers with a reduced jar exposing only the API interface. As inexperienced developers may interact directly with the core code causing unintended issues or bugs.

    \item The packages application must be run via the command line in its current stage, which limits accessibility to younger users and less technically inclined users. This should be rectified by producing an executable installer that installs and executes the application like consumers have come to expect.
\end{itemize}





\chapter{Conclusions}
\label{ch:conclusions}
The goal of the project was to produce a simulator for RISC-V \cite{riscv_2015_riscv} to provide a interactive platform to improve a users understanding of how a RISC-V processor works. The resulting application provides this functionality and more as a result of careful planning and execution with the additional modules system providing further extensibility beyond original considerations. As a result the project has been a success.

Within the project a full simulator has been built from the ground up from the concept of a emulator linking into a visualisation system. Then with the addition of the Module System providing a simple way to extend the system further. The project allows users to enter RISC-V \cite{riscv_2015_riscv} assembly code and visually observe how an implementation of the architecture may move data between the physical components and interact with the registers and memory. Whilst the project does have a few previously motioned limitations, these can all be resolved with future work and maintenance.

\section{Future work}\label{sec:future_work}
With the project being a simulator, there is no end of possibilities for future work in terms of adding additional visualisation options, as well as implementing the available RISC-V \cite{riscv_2015_riscv} extensions available.

Below, we'll discuss all the possible bits of future work:

\subsection{Implementing the Rest of the Base Instruction Set}
It would be ideal to complete the base instruction set implementation. Including the missing Control Status Flag, Synch, Environment and Jump \& Link instructions so that more complex programs can be written by more advanced users. Currently these instructions wont be recognised by the program. However, they can be added by a module in the future, or directly into the base program. 

\subsection{Additional Modules}
Currently other than the base instruction set, only 2 additional sets have been implemented: Multiply and Divide and Single Precision Floating Point. 

It would be beneficial to implement more of the extensions such as 64 bit integer support, double and quadruple precision floating point and more complex instruction sets such as vector operations and atomic instructions. 

Implementing these modules would provide more ways to interact with the application and provide more visualisations of how more complex instructions operate within the processor. However these may require additional tooling to implement with changes required to the module interface to implement.

\subsection{Labelled Looping and Stepping}
Both Labelled looping and Stepping are requirements that were never complete. Stepping would be ideal to implement to allow for users to go back and re-watch instructions without having to restart and run the whole animation sequence to get to the desired instruction.

Labelled looping would permit the ability to call blocks of code, allowing for more complex execution and the somewhat creation of named functions, rather than currently changing the program counter value by a relative amount.

\subsection{Enhanced Syntax and Lexical Feedback}
Currently the syntax and lexical feedback is limited, but effective. It would be nice to improve this to add highlighting to the code editor to allow for quicker identifying of issues, rather than relying on an alert box to inform the user.

This would aim to include highlighting affected rows, highlighting specific parts of rows, and underling invalid parts as most other integrated development environments do.

This addition would require a partial rework to the parser and major modifications to the code editor, but would be fruitful in providing feedback for novice RISC-V coders, allowing them to easily fix errors and not be pushed away by confusing or lacklustre error messages.




\appendix
\chapter{Timetable}\label{app:timetable}
\begin{figure}[h]
    \centering
    \includegraphics[width=1\textwidth]{dissertation/DATA/t1.png}
    \caption{Term 1 Timetable}
    \label{fig:tt_t1}
\end{figure}
\begin{figure}[h]
    \centering
    \includegraphics[width=1\textwidth]{dissertation/DATA/t2.png}
    \caption{Term 2 Timetable}
    \label{fig:tt_t2}
\end{figure}
\begin{figure}[h]
    \centering
    \includegraphics[width=0.8\textwidth]{dissertation/DATA/chr.png}
    \caption{Christmas Timetable}
    \label{fig:tt_christmas}
\end{figure}
\begin{figure}[h]
    \centering
    \includegraphics[width=0.8\textwidth]{dissertation/DATA/east.png}
    \caption{Easter Timetable}
    \label{fig:tt_easter}
\end{figure}

\chapter{Specification}
\includepdf[pages=-]{dissertation/DATA/specification.pdf}

\chapter{Progress Report}
\includepdf[pages=-]{dissertation/DATA/progress_report.pdf}

\printbibliography
\end{document}